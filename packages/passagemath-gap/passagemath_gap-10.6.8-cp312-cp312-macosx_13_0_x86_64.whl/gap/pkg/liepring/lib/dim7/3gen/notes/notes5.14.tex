%% This document created by Scientific Word (R) Version 2.5
%% Starting shell: mathart1


\documentclass[10pt,thmsa]{article}
%%%%%%%%%%%%%%%%%%%%%%%%%%%%%%%%%%%%%%%%%%%%%%%%%%%%%%%%%%%%%%%%%%%%%%%%%%%%%%%%%%%%%%%%%%%%%%%%%%%%%%%%%%%%%%%%%%%%%%%%%%%%%%%%%%%%%%%%%%%%%%%%%%%%%%%%%%%%%%%%%%%%%%%%%%%%%%%%%%%%%%%%%%%%%%%%%%%%%%%%%%%%%%%%%%%%%%%%%%%%%%%%%%%%%%%%%%%%%%%%%%%%%%%%%%%%
\usepackage{amsfonts}
\usepackage{amssymb}
\usepackage{sw20elba}

%TCIDATA{TCIstyle=article/art1.lat,elba,article}

%TCIDATA{OutputFilter=LATEX.DLL}
%TCIDATA{Version=5.50.0.2890}
%TCIDATA{<META NAME="SaveForMode" CONTENT="1">}
%TCIDATA{BibliographyScheme=Manual}
%TCIDATA{Created=Thu Nov 25 13:56:53 2004}
%TCIDATA{LastRevised=Tuesday, August 20, 2013 23:41:31}
%TCIDATA{<META NAME="GraphicsSave" CONTENT="32">}
%TCIDATA{Language=American English}

\input{tcilatex}
\begin{document}

\author{Michael Vaughan-Lee}
\title{Notes5.14}
\date{July 2013}
\maketitle

\section{Immediate descendants of algebra 5.14}

Algebra 5.14 has 
\begin{eqnarray*}
&&2p^{5}+7p^{4}+19p^{3}+49p^{2}+128p+256+(p^{2}+7p+29)\gcd (p-1,3) \\
&&\;\;\;+(p^{2}+7p+24)\gcd (p-1,4)+(p+3)\gcd (p-1,5)
\end{eqnarray*}%
immediate descendants of order $p^{7}$ and $p$-class $3$.

Algebra 5.14 has presentation 
\[
\langle a,b,c\,|\,cb,\,pa,\,pb,\,pc,\,\text{class }2\rangle , 
\]
and so if $L$ is an immediate descendant of 5.14 of order $p^7$ then $L_2$
is generated by $ba$, $ca$ modulo $L_3$, and $L_3$ has order $p^2$ and is
generated by $baa$, $bab$, $bac$, $caa$, $cab$. And $cb$, $pa$, $pb$, $pc\in
L_3$. The commutator structure is the same as one of 7.65 -- 7.88 from the
list of nilpotent Lie algebras over $\mathbb{Z}_p$. So we may assume that
one of the following holds:

\begin{equation}
cb=caa=cab=cac=0,  \tag{7.65}
\end{equation}%
\begin{equation}
caa=cab=cac=0,\,cb=baa,  \tag{7.66}
\end{equation}%
\begin{equation}
cb=bab=bac=cab=cac=0,  \tag{7.67}
\end{equation}%
\begin{equation}
cb=baa,\,bab=bac=cab=cac=0,  \tag{7.68}
\end{equation}%
\begin{equation}
cb=bac=cac=0,\,caa=bab,  \tag{7.69}
\end{equation}%
\begin{equation}
cb=baa,\,bac=cac=0,\,caa=bab,  \tag{7.70}
\end{equation}%
\begin{equation}
cb=baa=bac=cac=0,  \tag{7.71}
\end{equation}%
\begin{equation}
baa=bac=cac=0,\,cb=caa,  \tag{7.72}
\end{equation}%
\begin{equation}
cb=bac=caa=0,\,cac=bab,  \tag{7.73}
\end{equation}%
\begin{equation}
cb=bac=caa=0,\,cac=\omega bab,  \tag{7.74}
\end{equation}%
\begin{equation}
bac=caa=0,\,cb=baa,\,cac=bab,  \tag{7.75}
\end{equation}%
\begin{equation}
bac=caa=0,\,cb=baa,\,cac=\omega bab,  \tag{7.76}
\end{equation}%
\begin{equation}
cb=bac=0,\,caa=baa,\,cac=-bab,  \tag{7.77}
\end{equation}%
\begin{equation}
bac=0,\,cb=caa=baa,\,cac=-bab,  \tag{7.78}
\end{equation}%
\begin{equation}
cb=baa=bac=caa=0,  \tag{7.79}
\end{equation}%
\begin{equation}
cb=bac=caa=0,\,baa=cac,  \tag{7.80}
\end{equation}%
\begin{equation}
cb=bac=0,\,baa=cac,\,caa=bab,  \tag{7.81}
\end{equation}%
\begin{equation}
cb=bac=0,\,baa=cac,\,caa=\omega bab,\,(p=1\func{mod}3)  \tag{7.82}
\end{equation}%
\begin{equation}
cb=baa=caa=cac=0,  \tag{7.83}
\end{equation}%
\begin{equation}
cb=baa=cac=0,\,caa=bab,  \tag{7.84}
\end{equation}%
\begin{equation}
cb=caa=cac=0,\,baa=bab,  \tag{7.85}
\end{equation}%
\begin{equation}
cb=baa=caa=0,\,cac=\omega bab,  \tag{7.86}
\end{equation}%
\begin{equation}
cb=baa=0,\,caa=bac,\,cac=\omega bab,  \tag{7.87}
\end{equation}%
\begin{equation}
cb=baa=0,\,caa=kbab+bac,\,cac=\omega bab,\,(p=2\func{mod}3),  \tag{7.88}
\end{equation}%
where $k$ is any (one) integer which is not a value of 
\[
\frac{\lambda (\lambda ^{2}+3\omega \mu ^{2})}{\mu (3\lambda ^{2}+\omega \mu
^{2})}\func{mod}p. 
\]

Since the total number of descendants of 5.14 of order $p^{7}$ is of order $%
2p^{5}$, we need presentations with at least 5 parameters in some of these
cases. In each case the commutator structure is determined, and so to give a
presentation for the Lie rings we only need to specify $pa,pb,pc$. These
powers take values in $L_{3}$, which has order $p^{2}$, so we need 2
coefficients for each of $pa,pb,pc$. For the sake of simplicity I give a
single presentation with 6 parameters for each of the 24 different
commutator structures defined above, and I give the conditions for two sets
of parameters to define isomorphic Lie rings. In most of the cases I was
able to \textquotedblleft solve\textquotedblright\ the isomorphism problem
in the sense that I was able to produce a number of presentations with fewer
parameters, and with simple conditions on the parameters. But I was not able
to do this in every case.

The file notes5.14.m gives \textsc{Magma} programs to compute each
isomorphism class. The programs have complexity at most $p^{5}$, in the
sense that they have nested loops and the statements in the innermost loops
are executed a maximum of $O(p^{5})$ times. The programs run reasonably fast
for $p<20$, but you need to take a deep breath before running them for $p>20$%
. Apart from anything else the shear number of groups becomes overwhelming
pretty quickly. My classification of the nilpotent Lie rings of order $p^{7}$
has been criticized on the grounds that the Lie rings for any given prime
have to be computed \textquotedblleft on the fly\textquotedblright .
However, as I observed above, you need some presentations with at least 5
parameters, and even if you had five parameters independently taking all
values between $0$ and $p-1$ you would still need a program of complexity $%
p^{5}$ to print them out.

\subsection{Case 1}

\[
\langle
a,b,c\,|%
\,cb,caa,cab,cac,pa-x_{1}baa-x_{2}bab,pb-x_{3}baa-x_{4}bab,pc-x_{5}baa-x_{6}bab\rangle . 
\]

Here $L_{3}$ is generated by $baa$ and $bab$, and if we let 
\[
\left( 
\begin{array}{l}
pa \\ 
pb \\ 
pc%
\end{array}%
\right) =A\left( 
\begin{array}{l}
baa \\ 
bab%
\end{array}%
\right) 
\]%
then the isomorphism classes of algebras satisfying these commutator
relations correspond to the orbits of $3\times 2$ matrices $A$ under the
action 
\[
A\rightarrow \left( 
\begin{array}{lll}
\alpha & \beta & \gamma \\ 
0 & \lambda & \mu \\ 
0 & 0 & \xi%
\end{array}%
\right) A\left( 
\begin{array}{ll}
\alpha ^{2}\lambda & \alpha \beta \lambda \\ 
0 & \alpha \lambda ^{2}%
\end{array}%
\right) ^{-1}. 
\]

\bigskip 
\[
\left( 
\begin{array}{lll}
\alpha & \beta & \gamma \\ 
0 & \lambda & \mu \\ 
0 & 0 & \xi%
\end{array}%
\right) \left( 
\begin{array}{cc}
x_{1} & x_{2} \\ 
x_{3} & x_{4} \\ 
x_{5} & x_{6}%
\end{array}%
\right) \left( 
\begin{array}{ll}
\alpha ^{2}\lambda & \alpha \beta \lambda \\ 
0 & \alpha \lambda ^{2}%
\end{array}%
\right) ^{-1} 
\]%
\[
=\left( 
\begin{array}{cc}
\frac{1}{\alpha ^{2}\lambda }\left( \alpha x_{1}+\beta x_{3}+\gamma
x_{5}\right) & \frac{1}{\alpha \lambda ^{2}}\left( \alpha x_{2}+\beta
x_{4}+\gamma x_{6}\right) -\frac{1}{\alpha ^{2}}\frac{\beta }{\lambda ^{2}}%
\left( \alpha x_{1}+\beta x_{3}+\gamma x_{5}\right) \\ 
\frac{1}{\alpha ^{2}\lambda }\left( \lambda x_{3}+\mu x_{5}\right) & \frac{1%
}{\alpha \lambda ^{2}}\left( \lambda x_{4}+\mu x_{6}\right) -\frac{1}{\alpha
^{2}}\frac{\beta }{\lambda ^{2}}\left( \lambda x_{3}+\mu x_{5}\right) \\ 
\frac{1}{\alpha ^{2}\lambda }\xi x_{5} & \frac{1}{\alpha \lambda ^{2}}\xi
x_{6}-\frac{1}{\alpha ^{2}}\frac{\beta }{\lambda ^{2}}\xi x_{5}%
\end{array}%
\right) 
\]%
$\allowbreak $

There are $3p+22$ agebras in all in this case.

\subsection{Case 2}

\[
\langle
a,b,c\,|%
\,cb-baa,caa,cab,cac,pa-x_{1}baa-x_{2}bab,pb-x_{3}baa-x_{4}bab,pc-x_{5}baa-x_{6}bab\rangle . 
\]%
Here $L_{3}$ is generated by $baa$ and $bab$, and if we let 
\[
\left( 
\begin{array}{l}
pa \\ 
pb \\ 
pc%
\end{array}%
\right) =A\left( 
\begin{array}{l}
baa \\ 
bab%
\end{array}%
\right) 
\]%
then the isomorphism classes of algebras satisfying these commutator
relations correspond to the orbits of $3\times 2$ matrices $A$ under the
action 
\[
A\rightarrow \left( 
\begin{array}{lll}
\alpha & \beta & \gamma \\ 
0 & \lambda & \mu \\ 
0 & 0 & \alpha ^{2}%
\end{array}%
\right) A\left( 
\begin{array}{ll}
\alpha ^{2}\lambda & \alpha \beta \lambda \\ 
0 & \alpha \lambda ^{2}%
\end{array}%
\right) ^{-1}. 
\]

\[
\left( 
\begin{array}{lll}
\alpha & \beta & \gamma \\ 
0 & \lambda & \mu \\ 
0 & 0 & \alpha ^{2}%
\end{array}%
\right) \left( 
\begin{array}{cc}
x_{1} & x_{2} \\ 
x_{3} & x_{4} \\ 
x_{5} & x_{6}%
\end{array}%
\right) \left( 
\begin{array}{ll}
\alpha ^{2}\lambda & \alpha \beta \lambda \\ 
0 & \alpha \lambda ^{2}%
\end{array}%
\right) ^{-1} 
\]
\[
=\left( 
\begin{array}{cc}
\frac{1}{\alpha ^{2}\lambda }\left( \alpha x_{1}+\beta x_{3}+\gamma
x_{5}\right) & \frac{1}{\alpha \lambda ^{2}}\left( \alpha x_{2}+\beta
x_{4}+\gamma x_{6}\right) -\frac{1}{\alpha ^{2}}\frac{\beta }{\lambda ^{2}}%
\left( \alpha x_{1}+\beta x_{3}+\gamma x_{5}\right) \\ 
\frac{1}{\alpha ^{2}\lambda }\left( \lambda x_{3}+\mu x_{5}\right) & \frac{1%
}{\alpha \lambda ^{2}}\left( \lambda x_{4}+\mu x_{6}\right) -\frac{1}{\alpha
^{2}}\frac{\beta }{\lambda ^{2}}\left( \lambda x_{3}+\mu x_{5}\right) \\ 
\frac{1}{\lambda }x_{5} & \frac{\alpha }{\lambda ^{2}}x_{6}-\frac{\beta }{%
\lambda ^{2}}x_{5}%
\end{array}%
\right) \allowbreak . 
\]

The total number of algebras in this case is $5p+13+\gcd (p-1,3)+\gcd
(p-1,4) $.

\subsection{Case 3}

\[
\langle
a,b,c\,|%
\,cb,bab,bac,cab,cac,pa-x_{1}baa-x_{2}caa,pb-x_{3}baa-x_{4}caa,pc-x_{5}baa-x_{6}caa\rangle . 
\]%
$L_{3}$ is generated by $baa$ and $caa$ and if we let 
\[
\left( 
\begin{array}{l}
pa \\ 
pb \\ 
pc%
\end{array}%
\right) =A\left( 
\begin{array}{l}
baa \\ 
caa%
\end{array}%
\right) 
\]%
then the isomorphism classes of algebras satisfying these commutator
relations correspond to the orbits of $3\times 2$ matrices $A$ under the
action 
\[
A\rightarrow \left( 
\begin{array}{lll}
\alpha & \beta & \gamma \\ 
0 & \lambda & \mu \\ 
0 & \nu & \xi%
\end{array}%
\right) A\left( 
\begin{array}{ll}
\alpha ^{2}\lambda & \alpha ^{2}\mu \\ 
\alpha ^{2}\nu & \alpha ^{2}\xi%
\end{array}%
\right) ^{-1}. 
\]

\[
\left( 
\begin{array}{lll}
\alpha & \beta & \gamma \\ 
0 & \lambda & \mu \\ 
0 & \nu & \xi%
\end{array}%
\right) \left( 
\begin{array}{cc}
x_{1} & x_{2} \\ 
x_{3} & x_{4} \\ 
x_{5} & x_{6}%
\end{array}%
\right) \left( 
\begin{array}{ll}
\alpha ^{2}\lambda & \alpha ^{2}\mu \\ 
\alpha ^{2}\nu & \alpha ^{2}\xi%
\end{array}%
\right) ^{-1} 
\]

$\allowbreak $ $\allowbreak $

\begin{eqnarray*}
&=&\frac{1}{\alpha ^{2}\lambda \xi -\alpha ^{2}\mu \nu } \\
&&\times \left( 
\begin{array}{cc}
\alpha \xi x_{1}-\alpha \nu x_{2}-\beta \nu x_{4}+\beta \xi x_{3}-\gamma \nu
x_{6}+\gamma \xi x_{5} & \alpha \lambda x_{2}-\alpha \mu x_{1}+\beta \lambda
x_{4}-\beta \mu x_{3}+\lambda \gamma x_{6}-\gamma \mu x_{5} \\ 
\lambda \xi x_{3}-\lambda \nu x_{4}-\mu \nu x_{6}+\mu \xi x_{5} & \lambda
^{2}x_{4}-\mu ^{2}x_{5}-\lambda \mu x_{3}+\lambda \mu x_{6} \\ 
\xi ^{2}x_{5}-\nu ^{2}x_{4}+\nu \xi x_{3}-\nu \xi x_{6} & \lambda \nu
x_{4}-\mu \nu x_{3}+\lambda \xi x_{6}-\mu \xi x_{5}%
\end{array}%
\right)
\end{eqnarray*}%
$\allowbreak $

The total number of algebras in this case is $2p+8+\gcd (p-1,4)$.

\subsection{Case 4}

\[
\langle
a,b,c\,|%
\,cb-baa,bab,bac,cab,cac,pa-x_{1}baa-x_{2}caa,pb-x_{3}baa-x_{4}caa,pc-x_{5}baa-x_{6}caa\rangle . 
\]%
\[
cb=baa,\,bab=bac=cab=cac=0. 
\]%
$L_{3}$ is generated by $baa$ and $caa$ and if we let 
\[
\left( 
\begin{array}{l}
pa \\ 
pb \\ 
pc%
\end{array}%
\right) =A\left( 
\begin{array}{l}
baa \\ 
caa%
\end{array}%
\right) 
\]%
then the isomorphism classes of algebras satisfying these commutator
relations correspond to the orbits of $3\times 2$ matrices $A$ under the
action 
\[
A\rightarrow \left( 
\begin{array}{lll}
\alpha & \beta & \gamma \\ 
0 & \lambda & 0 \\ 
0 & \nu & \alpha ^{2}%
\end{array}%
\right) A\left( 
\begin{array}{ll}
\alpha ^{2}\lambda & 0 \\ 
\alpha ^{2}\nu & \alpha ^{4}%
\end{array}%
\right) ^{-1}. 
\]

\[
\left( 
\begin{array}{lll}
\alpha & \beta & \gamma \\ 
0 & \lambda & 0 \\ 
0 & \nu & \alpha ^{2}%
\end{array}%
\right) \left( 
\begin{array}{cc}
x_{1} & x_{2} \\ 
x_{3} & x_{4} \\ 
x_{5} & x_{6}%
\end{array}%
\right) \left( 
\begin{array}{ll}
\alpha ^{2}\lambda & 0 \\ 
\alpha ^{2}\nu & \alpha ^{4}%
\end{array}%
\right) ^{-1} 
\]%
$\allowbreak \allowbreak $

\[
=\frac{1}{\alpha ^{4}}\allowbreak \left( 
\begin{array}{cc}
\frac{1}{\lambda }\left( \alpha ^{3}x_{1}+\alpha ^{2}\beta x_{3}+\alpha
^{2}\gamma x_{5}-\alpha \nu x_{2}-\beta \nu x_{4}-\gamma \nu x_{6}\right) & 
\alpha x_{2}+\beta x_{4}+\gamma x_{6} \\ 
\alpha ^{2}x_{3}-\nu x_{4} & \lambda x_{4} \\ 
\frac{1}{\lambda }\left( \alpha ^{4}x_{5}-\nu ^{2}x_{4}+\alpha ^{2}\nu
x_{3}-\alpha ^{2}\nu x_{6}\right) & x_{6}\alpha ^{2}+\nu x_{4}%
\end{array}%
\right) 
\]%
$\allowbreak $

The total number of algebras in this case is $6p+8+2\gcd (p-1,3)+\gcd
(p-1,4)+\gcd (p-1,5)$.

\subsection{Case 5}

\[
\langle
a,b,c\,|%
\,cb,bac,caa-bab,cac,pa-x_{1}baa-x_{2}bab,pb-x_{3}baa-x_{4}bab,pc-x_{5}baa-x_{6}bab\rangle . 
\]%
$L_{3}$ is generated by $baa$ and $bab$ and if we let 
\[
\left( 
\begin{array}{l}
pa \\ 
pb \\ 
pc%
\end{array}%
\right) =A\left( 
\begin{array}{l}
baa \\ 
bab%
\end{array}%
\right) 
\]%
then the isomorphism classes of algebras satisfying these commutator
relations correspond to the orbits of $3\times 2$ matrices $A$ under the
action 
\[
A\rightarrow \left( 
\begin{array}{lll}
\alpha & \beta & \gamma \\ 
0 & \lambda & \mu \\ 
0 & 0 & \alpha ^{-1}\lambda ^{2}%
\end{array}%
\right) A\left( 
\begin{array}{ll}
\alpha ^{2}\lambda & \alpha ^{2}\mu +\alpha \beta \lambda \\ 
0 & \alpha \lambda ^{2}%
\end{array}%
\right) ^{-1}. 
\]

\[
\left( 
\begin{array}{lll}
\alpha & \beta & \gamma \\ 
0 & \lambda & \mu \\ 
0 & 0 & \alpha ^{-1}\lambda ^{2}%
\end{array}%
\right) \left( 
\begin{array}{cc}
x_{1} & x_{2} \\ 
x_{3} & x_{4} \\ 
x_{5} & x_{6}%
\end{array}%
\right) \left( 
\begin{array}{ll}
\alpha ^{2}\lambda & \alpha ^{2}\mu +\alpha \beta \lambda \\ 
0 & \alpha \lambda ^{2}%
\end{array}%
\right) ^{-1} 
\]
\begin{eqnarray*}
&=&\frac{1}{\alpha ^{2}\lambda ^{3}}\times \\
&&\left( 
\begin{array}{cc}
\lambda ^{2}\left( \alpha x_{1}+\beta x_{3}+\gamma x_{5}\right) & \alpha
^{2}\lambda x_{2}-\alpha ^{2}\mu x_{1}-\beta ^{2}\lambda x_{3}-\alpha \beta
\lambda x_{1}+\alpha \beta \lambda x_{4}-\alpha \beta \mu x_{3}+\alpha
\lambda \gamma x_{6}-\alpha \gamma \mu x_{5}-\beta \lambda \gamma x_{5} \\ 
\lambda ^{2}\left( \lambda x_{3}+\mu x_{5}\right) & \alpha \lambda
^{2}x_{4}-\beta \lambda ^{2}x_{3}-\alpha \mu ^{2}x_{5}-\alpha \lambda \mu
x_{3}+\alpha \lambda \mu x_{6}-\beta \lambda \mu x_{5} \\ 
\frac{1}{\alpha }\lambda ^{4}x_{5} & -\frac{1}{\alpha }\lambda ^{2}\left(
\alpha \mu x_{5}-\alpha \lambda x_{6}+\beta \lambda x_{5}\right)%
\end{array}%
\right)
\end{eqnarray*}%
$\allowbreak \allowbreak $

The total number of algebras in this case is $5p+13+2\gcd (p-1,3)+\gcd
(p-1,4)$.

\subsection{Case 6}

\[
\langle
a,b,c\,|%
\,cb-baa,bac,caa-bab,cac,pa-x_{1}baa-x_{2}bab,pb-x_{3}baa-x_{4}bab,pc-x_{5}baa-x_{6}bab\rangle . 
\]%
$L_{3}$ is generated by $baa$ and $bab$ and if we let 
\[
\left( 
\begin{array}{l}
pa \\ 
pb \\ 
pc%
\end{array}%
\right) =A\left( 
\begin{array}{l}
baa \\ 
bab%
\end{array}%
\right) 
\]%
then the isomorphism classes of algebras satisfying these commutator
relations correspond to the orbits of $3\times 2$ matrices $A$ under the
action 
\[
A\rightarrow \left( 
\begin{array}{lll}
\alpha ^{2} & \beta & \gamma \\ 
0 & \pm \alpha ^{3} & \mu \\ 
0 & 0 & \alpha ^{4}%
\end{array}%
\right) A\left( 
\begin{array}{ll}
\pm \alpha ^{7} & \alpha ^{4}\mu \pm \alpha ^{5}\beta \\ 
0 & \alpha ^{8}%
\end{array}%
\right) ^{-1}. 
\]

\[
\left( 
\begin{array}{lll}
\alpha ^{2} & \beta & \gamma \\ 
0 & \alpha ^{3} & \mu \\ 
0 & 0 & \alpha ^{4}%
\end{array}%
\right) \left( 
\begin{array}{cc}
x_{1} & x_{2} \\ 
x_{3} & x_{4} \\ 
x_{5} & x_{6}%
\end{array}%
\right) \left( 
\begin{array}{ll}
\alpha ^{7} & \alpha ^{4}\mu +\alpha ^{5}\beta \\ 
0 & \alpha ^{8}%
\end{array}%
\right) ^{-1} 
\]
\[
=\left( 
\begin{array}{cc}
\frac{1}{\alpha ^{7}}\left( x_{1}\alpha ^{2}+\beta x_{3}+\gamma x_{5}\right)
& \frac{1}{\alpha ^{8}}\left( x_{2}\alpha ^{2}+\beta x_{4}+\gamma
x_{6}\right) -\frac{1}{\alpha ^{11}}\left( \mu +\alpha \beta \right) \left(
x_{1}\alpha ^{2}+\beta x_{3}+\gamma x_{5}\right) \\ 
\frac{1}{\alpha ^{7}}\left( x_{3}\alpha ^{3}+\mu x_{5}\right) & \frac{1}{%
\alpha ^{8}}\left( x_{4}\alpha ^{3}+\mu x_{6}\right) -\frac{1}{\alpha ^{11}}%
\left( \mu +\alpha \beta \right) \left( x_{3}\alpha ^{3}+\mu x_{5}\right) \\ 
\frac{1}{\alpha ^{3}}x_{5} & \frac{1}{\alpha ^{4}}x_{6}-\frac{1}{\alpha ^{7}}%
x_{5}\left( \mu +\alpha \beta \right)%
\end{array}%
\right) \allowbreak , 
\]%
\[
\left( 
\begin{array}{lll}
\alpha ^{2} & \beta & \gamma \\ 
0 & -\alpha ^{3} & \mu \\ 
0 & 0 & \alpha ^{4}%
\end{array}%
\right) \left( 
\begin{array}{cc}
x_{1} & x_{2} \\ 
x_{3} & x_{4} \\ 
x_{5} & x_{6}%
\end{array}%
\right) \left( 
\begin{array}{ll}
-\alpha ^{7} & \alpha ^{4}\mu -\alpha ^{5}\beta \\ 
0 & \alpha ^{8}%
\end{array}%
\right) ^{-1} 
\]
\[
=\left( 
\begin{array}{cc}
-\frac{1}{\alpha ^{7}}\left( x_{1}\alpha ^{2}+\beta x_{3}+\gamma x_{5}\right)
& \frac{1}{\alpha ^{8}}\left( x_{2}\alpha ^{2}+\beta x_{4}+\gamma
x_{6}\right) +\frac{1}{\alpha ^{11}}\left( \mu -\alpha \beta \right) \left(
x_{1}\alpha ^{2}+\beta x_{3}+\gamma x_{5}\right) \\ 
-\frac{1}{\alpha ^{7}}\left( \mu x_{5}-\alpha ^{3}x_{3}\right) & \frac{1}{%
\alpha ^{8}}\left( \mu x_{6}-\alpha ^{3}x_{4}\right) +\frac{1}{\alpha ^{11}}%
\left( \mu -\alpha \beta \right) \left( \mu x_{5}-\alpha ^{3}x_{3}\right) \\ 
-\frac{1}{\alpha ^{3}}x_{5} & \frac{1}{\alpha ^{4}}x_{6}+\frac{1}{\alpha ^{7}%
}x_{5}\left( \mu -\alpha \beta \right)%
\end{array}%
\right) 
\]%
$\allowbreak $

The total number of algebras in this case is 
\[
p^2+3p-3+(p+2)\gcd (p-1,3)+(p+1)\gcd (p-1,4)+(p+1)\gcd (p-1,5). 
\]

\subsection{Case 7}

\[
\langle
a,b,c\,|%
\,cb,baa,bac,cac,pa-x_{1}bab-x_{2}caa,pb-x_{3}bab-x_{4}caa,pc-x_{5}bab-x_{6}caa\rangle . 
\]%
$L_{3}$ is generated by $bab$ and $caa$ and if we let 
\[
\left( 
\begin{array}{l}
pa \\ 
pb \\ 
pc%
\end{array}%
\right) =A\left( 
\begin{array}{l}
bab \\ 
caa%
\end{array}%
\right) 
\]%
then the isomorphism classes of algebras satisfying these commutator
relations correspond to the orbits of $3\times 2$ matrices $A$ under the
action 
\[
A\rightarrow \left( 
\begin{array}{lll}
\alpha & 0 & \gamma \\ 
0 & \lambda & 0 \\ 
0 & 0 & \xi%
\end{array}%
\right) A\left( 
\begin{array}{ll}
\alpha \lambda ^{2} & 0 \\ 
0 & \alpha ^{2}\xi%
\end{array}%
\right) ^{-1}. 
\]%
Now 
\[
\left( 
\begin{array}{lll}
\alpha & 0 & \gamma \\ 
0 & \lambda & 0 \\ 
0 & 0 & \xi%
\end{array}%
\right) \left( 
\begin{array}{cc}
x_{1} & x_{2} \\ 
x_{3} & x_{4} \\ 
x_{5} & x_{6}%
\end{array}%
\right) \left( 
\begin{array}{ll}
\alpha \lambda ^{2} & 0 \\ 
0 & \alpha ^{2}\xi%
\end{array}%
\right) ^{-1} 
\]
\[
=\left( 
\begin{array}{cc}
\frac{1}{\alpha \lambda ^{2}}\left( \alpha x_{1}+\gamma x_{5}\right) & \frac{%
1}{\alpha ^{2}\xi }\left( \alpha x_{2}+\gamma x_{6}\right) \\ 
\frac{1}{\alpha \lambda }x_{3} & \frac{1}{\alpha ^{2}}\frac{\lambda }{\xi }%
x_{4} \\ 
\frac{1}{\alpha \lambda ^{2}}\xi x_{5} & \frac{1}{\alpha ^{2}}x_{6}%
\end{array}%
\right) . 
\]%
$\allowbreak $

The total number of algebras in this case is $2p^2+11p+43+\gcd (p-1,4)$.

\subsection{Case 8}

\[
\langle
a,b,c\,|%
\,cb-caa,baa,bac,cac,pa-x_{1}bab-x_{2}caa,pb-x_{3}bab-x_{4}caa,pc-x_{5}bab-x_{6}caa\rangle . 
\]%
$L_{3}$ is generated by $bab$ and $caa$ and if we let 
\[
\left( 
\begin{array}{l}
pa \\ 
pb \\ 
pc%
\end{array}%
\right) =A\left( 
\begin{array}{l}
bab \\ 
caa%
\end{array}%
\right) 
\]%
then the isomorphism classes of algebras satisfying these commutator
relations correspond to the orbits of $3\times 2$ matrices $A$ under the
action 
\[
A\rightarrow \left( 
\begin{array}{lll}
\alpha & 0 & \gamma \\ 
0 & \alpha ^{2} & 0 \\ 
0 & 0 & \xi%
\end{array}%
\right) A\left( 
\begin{array}{ll}
\alpha ^{5} & 0 \\ 
0 & \alpha ^{2}\xi%
\end{array}%
\right) ^{-1}. 
\]%
Now 
\[
\left( 
\begin{array}{lll}
\alpha & 0 & \gamma \\ 
0 & \alpha ^{2} & 0 \\ 
0 & 0 & \xi%
\end{array}%
\right) \left( 
\begin{array}{cc}
x_{1} & x_{2} \\ 
x_{3} & x_{4} \\ 
x_{5} & x_{6}%
\end{array}%
\right) \left( 
\begin{array}{ll}
\alpha ^{5} & 0 \\ 
0 & \alpha ^{2}\xi%
\end{array}%
\right) ^{-1} 
\]
\[
=\left( 
\begin{array}{cc}
\frac{1}{\alpha ^{5}}\left( \alpha x_{1}+\gamma x_{5}\right) & \frac{1}{%
\alpha ^{2}\xi }\left( \alpha x_{2}+\gamma x_{6}\right) \\ 
\frac{1}{\alpha ^{3}}x_{3} & \frac{1}{\xi }x_{4} \\ 
\frac{1}{\alpha ^{5}}\xi x_{5} & \frac{1}{\alpha ^{2}}x_{6}%
\end{array}%
\right) . 
\]%
$\allowbreak $

The total number of algebras in this case is 
\[
p^3+4p^2+6p+(p+5)\gcd (p-1,3)+3\gcd (p-1,4)+\gcd (p-1,5). 
\]

\subsection{Case 9}

\[
\langle
a,b,c\,|%
\,cb,bac,caa,cac-bab,pa-x_{1}baa-x_{2}bab,pb-x_{3}baa-x_{4}bab,pc-x_{5}baa-x_{6}bab\rangle . 
\]%
$L_{3}$ is generated by $baa$ and $bab$ and if we let 
\[
\left( 
\begin{array}{l}
pa \\ 
pb \\ 
pc%
\end{array}%
\right) =A\left( 
\begin{array}{l}
baa \\ 
bab%
\end{array}%
\right) 
\]%
then the isomorphism classes of algebras satisfying these commutator
relations correspond to the orbits of $3\times 2$ matrices $A$ under the
action 
\[
A\rightarrow \left( 
\begin{array}{lll}
\alpha & \beta & 0 \\ 
0 & \lambda & 0 \\ 
0 & 0 & \pm \lambda%
\end{array}%
\right) A\left( 
\begin{array}{ll}
\alpha ^{2}\lambda & \alpha \beta \lambda \\ 
0 & \alpha \lambda ^{2}%
\end{array}%
\right) ^{-1}. 
\]

\[
\left( 
\begin{array}{lll}
\alpha & \beta & 0 \\ 
0 & \lambda & 0 \\ 
0 & 0 & \lambda%
\end{array}%
\right) \left( 
\begin{array}{cc}
x_{1} & x_{2} \\ 
x_{3} & x_{4} \\ 
x_{5} & x_{6}%
\end{array}%
\right) \left( 
\begin{array}{ll}
\alpha ^{2}\lambda & \alpha \beta \lambda \\ 
0 & \alpha \lambda ^{2}%
\end{array}%
\right) ^{-1} 
\]
\[
=\left( 
\begin{array}{cc}
\frac{1}{\alpha ^{2}\lambda }\left( \alpha x_{1}+\beta x_{3}\right) & \frac{1%
}{\alpha \lambda ^{2}}\left( \alpha x_{2}+\beta x_{4}\right) -\frac{1}{%
\alpha ^{2}}\frac{\beta }{\lambda ^{2}}\left( \alpha x_{1}+\beta x_{3}\right)
\\ 
\frac{1}{\alpha ^{2}}x_{3} & \frac{1}{\alpha \lambda }x_{4}-\frac{1}{\alpha
^{2}}\frac{\beta }{\lambda }x_{3} \\ 
\frac{1}{\alpha ^{2}}x_{5} & \frac{1}{\alpha \lambda }x_{6}-\frac{1}{\alpha
^{2}}\frac{\beta }{\lambda }x_{5}%
\end{array}%
\right) , 
\]

\[
\left( 
\begin{array}{lll}
\alpha & \beta & 0 \\ 
0 & \lambda & 0 \\ 
0 & 0 & -\lambda%
\end{array}%
\right) \left( 
\begin{array}{cc}
x_{1} & x_{2} \\ 
x_{3} & x_{4} \\ 
x_{5} & x_{6}%
\end{array}%
\right) \left( 
\begin{array}{ll}
\alpha ^{2}\lambda & \alpha \beta \lambda \\ 
0 & \alpha \lambda ^{2}%
\end{array}%
\right) ^{-1} 
\]
\[
\left( 
\begin{array}{cc}
\frac{1}{\alpha ^{2}\lambda }\left( \alpha x_{1}+\beta x_{3}\right) & \frac{1%
}{\alpha \lambda ^{2}}\left( \alpha x_{2}+\beta x_{4}\right) -\frac{1}{%
\alpha ^{2}}\frac{\beta }{\lambda ^{2}}\left( \alpha x_{1}+\beta x_{3}\right)
\\ 
\frac{1}{\alpha ^{2}}x_{3} & \frac{1}{\alpha \lambda }x_{4}-\frac{1}{\alpha
^{2}}\frac{\beta }{\lambda }x_{3} \\ 
-\frac{1}{\alpha ^{2}}x_{5} & \frac{1}{\alpha ^{2}}\frac{\beta }{\lambda }%
x_{5}-\frac{1}{\alpha \lambda }x_{6}%
\end{array}%
\right) = 
\]%
$\allowbreak \allowbreak $

The total number of algebras in this case is 
\[
p^3+\frac 52p^2+7p+\frac{19}2+\frac{p+4}2\gcd (p-1,4). 
\]

\subsection{Case 10}

\[
\langle a,b,c\,|\,cb,bac,caa,cac-\omega
bab,pa-x_{1}baa-x_{2}bab,pb-x_{3}baa-x_{4}bab,pc-x_{5}baa-x_{6}bab\rangle . 
\]%
$L_{3}$ is generated by $baa$ and $bab$ and if we let 
\[
\left( 
\begin{array}{l}
pa \\ 
pb \\ 
pc%
\end{array}%
\right) =A\left( 
\begin{array}{l}
baa \\ 
bab%
\end{array}%
\right) 
\]%
then the isomorphism classes of algebras satisfying these commutator
relations correspond to the orbits of $3\times 2$ matrices $A$ under the
action 
\[
A\rightarrow \left( 
\begin{array}{lll}
\alpha & \beta & 0 \\ 
0 & \lambda & 0 \\ 
0 & 0 & \pm \lambda%
\end{array}%
\right) A\left( 
\begin{array}{ll}
\alpha ^{2}\lambda & \alpha \beta \lambda \\ 
0 & \alpha \lambda ^{2}%
\end{array}%
\right) ^{-1}. 
\]

This case is identical to Case 9 and so there are 
\[
p^{3}+\frac{5}{2}p^{2}+7p+\frac{19}{2}+\frac{p+4}{2}\gcd (p-1,4) 
\]%
algebras here.

\subsection{Case 11}

\[
\langle
a,b,c\,|%
\,cb-baa,bac,caa,cac-bab,pa-x_{1}baa-x_{2}bab,pb-x_{3}baa-x_{4}bab,pc-x_{5}baa-x_{6}bab\rangle . 
\]%
$L_{3}$ is generated by $baa$ and $bab$ and if we let 
\[
\left( 
\begin{array}{l}
pa \\ 
pb \\ 
pc%
\end{array}%
\right) =A\left( 
\begin{array}{l}
baa \\ 
bab%
\end{array}%
\right) 
\]%
then the isomorphism classes of algebras satisfying these commutator
relations correspond to the orbits of $3\times 2$ matrices $A$ under the
action 
\[
A\rightarrow \left( 
\begin{array}{lll}
\alpha & \beta & 0 \\ 
0 & \pm \alpha ^{2} & 0 \\ 
0 & 0 & \alpha ^{2}%
\end{array}%
\right) A\left( 
\begin{array}{ll}
\pm \alpha ^{4} & \pm \alpha ^{3}\beta \\ 
0 & \alpha ^{5}%
\end{array}%
\right) ^{-1}. 
\]

\[
\left( 
\begin{array}{lll}
\alpha & \beta & 0 \\ 
0 & \alpha ^{2} & 0 \\ 
0 & 0 & \alpha ^{2}%
\end{array}%
\right) \left( 
\begin{array}{cc}
x_{1} & x_{2} \\ 
x_{3} & x_{4} \\ 
x_{5} & x_{6}%
\end{array}%
\right) \left( 
\begin{array}{ll}
\alpha ^{4} & \alpha ^{3}\beta \\ 
0 & \alpha ^{5}%
\end{array}%
\right) ^{-1} 
\]
\[
=\left( 
\begin{array}{cc}
\frac{1}{\alpha ^{4}}\left( \alpha x_{1}+\beta x_{3}\right) & \frac{1}{%
\alpha ^{5}}\left( \alpha x_{2}+\beta x_{4}\right) -\frac{1}{\alpha ^{6}}%
\beta \left( \alpha x_{1}+\beta x_{3}\right) \\ 
\frac{1}{\alpha ^{2}}x_{3} & \frac{1}{\alpha ^{3}}x_{4}-\frac{1}{\alpha ^{4}}%
\beta x_{3} \\ 
\frac{1}{\alpha ^{2}}x_{5} & \frac{1}{\alpha ^{3}}x_{6}-\frac{1}{\alpha ^{4}}%
\beta x_{5}%
\end{array}%
\right) , 
\]

\[
\left( 
\begin{array}{lll}
\alpha & \beta & 0 \\ 
0 & -\alpha ^{2} & 0 \\ 
0 & 0 & \alpha ^{2}%
\end{array}%
\right) \left( 
\begin{array}{cc}
x_{1} & x_{2} \\ 
x_{3} & x_{4} \\ 
x_{5} & x_{6}%
\end{array}%
\right) \left( 
\begin{array}{ll}
-\alpha ^{4} & -\alpha ^{3}\beta \\ 
0 & \alpha ^{5}%
\end{array}%
\right) ^{-1} 
\]
\[
=\left( 
\begin{array}{cc}
-\frac{1}{\alpha ^{4}}\left( \alpha x_{1}+\beta x_{3}\right) & \frac{1}{%
\alpha ^{5}}\left( \alpha x_{2}+\beta x_{4}\right) -\frac{1}{\alpha ^{6}}%
\beta \left( \alpha x_{1}+\beta x_{3}\right) \\ 
\frac{1}{\alpha ^{2}}x_{3} & \frac{1}{\alpha ^{4}}\beta x_{3}-\frac{1}{%
\alpha ^{3}}x_{4} \\ 
-\frac{1}{\alpha ^{2}}x_{5} & \frac{1}{\alpha ^{3}}x_{6}-\frac{1}{\alpha ^{4}%
}\beta x_{5}%
\end{array}%
\right) . 
\]%
$\allowbreak \allowbreak $

The total number of algebras in this case is 
\[
\allowbreak (p^{4}+p^{3}+4p^{2}+p-1+\allowbreak (p^{2}+2p+3)\gcd
(p-1,3)+(p+2)\gcd (p-1,4))/2 
\]

\subsection{Case 12}

\[
\langle a,b,c\,|\,cb-baa,bac,caa,cac-\omega
bab,pa-x_{1}baa-x_{2}bab,pb-x_{3}baa-x_{4}bab,pc-x_{5}baa-x_{6}bab\rangle . 
\]

This case is identical to Case 11, so again there are 
\[
\allowbreak (p^4+p^3+4p^2+p-1+\allowbreak (p^2+2p+3)\gcd (p-1,3)+(p+2)\gcd
(p-1,4))/2 
\]
algebras here.

\subsection{Case 13}

\[
\langle
a,b,c\,|%
\,cb,bac,caa-baa,cac+bab,pa-x_{1}baa-x_{2}bab,pb-x_{3}baa-x_{4}bab,pc-x_{5}baa-x_{6}bab\rangle . 
\]

$L_{3}$ is generated by $baa$ and $bab$ and if we let 
\[
\left( 
\begin{array}{l}
pa \\ 
pb \\ 
pc%
\end{array}%
\right) =A\left( 
\begin{array}{l}
baa \\ 
bab%
\end{array}%
\right) 
\]%
then the isomorphism classes of algebras satisfying these commutator
relations correspond to the orbits of $3\times 2$ matrices $A$ under the
action 
\[
A\rightarrow \left( 
\begin{array}{lll}
\alpha & \beta & -\beta \\ 
0 & \lambda & \mu \\ 
0 & \mu & \lambda%
\end{array}%
\right) A\left( 
\begin{array}{ll}
\alpha ^{2}(\lambda +\mu ) & \alpha \beta (\lambda +\mu ) \\ 
0 & \alpha (\lambda ^{2}-\mu ^{2})%
\end{array}%
\right) ^{-1}. 
\]

\[
\left( 
\begin{array}{lll}
\alpha & \beta & -\beta \\ 
0 & \lambda & \mu \\ 
0 & \mu & \lambda%
\end{array}%
\right) \left( 
\begin{array}{cc}
x_{1} & x_{2} \\ 
x_{3} & x_{4} \\ 
x_{5} & x_{6}%
\end{array}%
\right) \left( 
\begin{array}{ll}
\alpha ^{2}(\lambda +\mu ) & \alpha \beta (\lambda +\mu ) \\ 
0 & \alpha (\lambda ^{2}-\mu ^{2})%
\end{array}%
\right) ^{-1} 
\]
\[
=\frac{1}{\alpha ^{2}\lambda ^{2}-\alpha ^{2}\mu ^{2}}\left( 
\begin{array}{cc}
\left( \lambda -\mu \right) \left( \alpha x_{1}+\beta x_{3}-\beta
x_{5}\right) & \alpha ^{2}x_{2}-\beta ^{2}x_{3}+\beta ^{2}x_{5}-\alpha \beta
x_{1}+\alpha \beta x_{4}-\alpha \beta x_{6} \\ 
\left( \lambda -\mu \right) \left( \lambda x_{3}+\mu x_{5}\right) & \alpha
\lambda x_{4}-\beta \lambda x_{3}+\alpha \mu x_{6}-\beta \mu x_{5} \\ 
\left( \lambda -\mu \right) \left( \mu x_{3}+\lambda x_{5}\right) & \alpha
\mu x_{4}-\beta \mu x_{3}+\alpha \lambda x_{6}-\beta \lambda x_{5}%
\end{array}%
\right) . 
\]%
$\allowbreak \allowbreak $

In this case there are $2p^{2}+11p+27+\gcd (p-1,4)$ immediate descendants of
order $p^{7}$ and $p$-class 3.

\subsection{Case 14}

\[
\langle
a,b,c\,|%
\,cb-baa,bac,caa-baa,cac+bab,pa-x_{1}baa-x_{2}bab,pb-x_{3}baa-x_{4}bab,pc-x_{5}baa-x_{6}bab\rangle . 
\]%
$L_{3}$ is generated by $baa$ and $bab$ and if we let 
\[
\left( 
\begin{array}{l}
pa \\ 
pb \\ 
pc%
\end{array}%
\right) =A\left( 
\begin{array}{l}
baa \\ 
bab%
\end{array}%
\right) 
\]%
then the isomorphism classes of algebras satisfying these commutator
relations correspond to the orbits of $3\times 2$ matrices $A$ under the
action 
\[
A\rightarrow \left( 
\begin{array}{lll}
\alpha & \beta & -\beta \\ 
0 & \lambda & \lambda -\alpha ^{2} \\ 
0 & \lambda -\alpha ^{2} & \lambda%
\end{array}%
\right) A\left( 
\begin{array}{ll}
2\alpha ^{2}\lambda -\alpha ^{4} & 2\alpha \beta \lambda -\alpha ^{3}\beta
\\ 
0 & 2\alpha ^{3}\lambda -\alpha ^{5}%
\end{array}%
\right) ^{-1}. 
\]%
$\allowbreak \allowbreak $

In this case there are $p^{3}+2p^{2}+6p+10+(p+4)\gcd (p-1,3)$ algebras.

\subsection{Case 15}

\[
\langle
a,b,c\,|%
\,cb,baa,bac,caa,pa-x_{1}bab-x_{2}cac,pb-x_{3}bab-x_{4}cac,pc-x_{5}bab-x_{6}cac\rangle . 
\]%
$L_{3}$ is generated by $bab$ and $cac$ and if we let 
\[
\left( 
\begin{array}{l}
pa \\ 
pb \\ 
pc%
\end{array}%
\right) =A\left( 
\begin{array}{l}
bab \\ 
cac%
\end{array}%
\right) 
\]%
then the isomorphism classes of algebras satisfying these commutator
relations correspond to the orbits of $3\times 2$ matrices $A$ under the
action 
\[
A\rightarrow \left( 
\begin{array}{lll}
\alpha & 0 & 0 \\ 
0 & \beta & 0 \\ 
0 & 0 & \gamma%
\end{array}%
\right) A\left( 
\begin{array}{ll}
\alpha \beta ^{2} & 0 \\ 
0 & \alpha \gamma ^{2}%
\end{array}%
\right) ^{-1} 
\]%
and 
\[
A\rightarrow \left( 
\begin{array}{lll}
\alpha & 0 & 0 \\ 
0 & 0 & \beta \\ 
0 & \gamma & 0%
\end{array}%
\right) A\left( 
\begin{array}{ll}
0 & \alpha \beta ^{2} \\ 
\alpha \gamma ^{2} & 0%
\end{array}%
\right) ^{-1}. 
\]

\[
\left( 
\begin{array}{lll}
\alpha & 0 & 0 \\ 
0 & \beta & 0 \\ 
0 & 0 & \gamma%
\end{array}
\right) \left( 
\begin{array}{ll}
x & y \\ 
z & t \\ 
u & v%
\end{array}
\right) \left( 
\begin{array}{ll}
\alpha \beta ^2 & 0 \\ 
0 & \alpha \gamma ^2%
\end{array}
\right) ^{-1}=\allowbreak \left( 
\begin{array}{cc}
\frac x{\beta ^2} & \frac y{\gamma ^2} \\ 
\frac 1\beta \frac z\alpha & \beta \frac t{\alpha \gamma ^2} \\ 
\gamma \frac u{\alpha \beta ^2} & \frac 1\gamma \frac v\alpha%
\end{array}
\right) 
\]
\[
\left( 
\begin{array}{lll}
\alpha & 0 & 0 \\ 
0 & 0 & \beta \\ 
0 & \gamma & 0%
\end{array}
\right) \left( 
\begin{array}{ll}
x & y \\ 
z & t \\ 
u & v%
\end{array}
\right) \left( 
\begin{array}{ll}
0 & \alpha \beta ^2 \\ 
\alpha \gamma ^2 & 0%
\end{array}
\right) ^{-1}=\allowbreak \left( 
\begin{array}{cc}
\frac y{\beta ^2} & \frac x{\gamma ^2} \\ 
\frac 1\beta \frac v\alpha & \beta \frac u{\alpha \gamma ^2} \\ 
\gamma \frac t{\alpha \beta ^2} & \frac 1\gamma \frac z\alpha%
\end{array}
\right) 
\]

The total number of algebras in this case is 
\[
\allowbreak p^{3}+\frac{7}{2}p^{2}+\frac{17}{2}p+\frac{59}{2}+\frac{5}{2}%
\gcd (p-1,3)+\frac{p+1}{2}\gcd (p-1,4) 
\]

\subsection{Case 16}

\[
\langle
a,b,c\,|%
\,cb,bac,caa,cac-baa,pa-x_{1}baa-x_{2}bab,pb-x_{3}baa-x_{4}bab,pc-x_{5}baa-x_{6}bab\rangle . 
\]%
$L_{3}$ is generated by $baa$ and $bab$ and if we let 
\[
\left( 
\begin{array}{l}
pa \\ 
pb \\ 
pc%
\end{array}%
\right) =A\left( 
\begin{array}{l}
baa \\ 
bab%
\end{array}%
\right) 
\]%
then the isomorphism classes of algebras satisfying these commutator
relations correspond to the orbits of $3\times 2$ matrices $A$ under the
action 
\[
A\rightarrow \left( 
\begin{array}{lll}
\alpha & 0 & 0 \\ 
0 & \alpha ^{-1}\gamma ^{2} & 0 \\ 
0 & 0 & \gamma%
\end{array}%
\right) A\left( 
\begin{array}{ll}
\alpha \gamma ^{2} & 0 \\ 
0 & \alpha ^{-1}\gamma ^{4}%
\end{array}%
\right) ^{-1}. 
\]

\[
\left( 
\begin{array}{lll}
\alpha & 0 & 0 \\ 
0 & \alpha ^{-1}\gamma ^2 & 0 \\ 
0 & 0 & \gamma%
\end{array}
\right) \left( 
\begin{array}{ll}
x & y \\ 
z & t \\ 
u & v%
\end{array}
\right) \left( 
\begin{array}{ll}
\alpha \gamma ^2 & 0 \\ 
0 & \alpha ^{-1}\gamma ^4%
\end{array}
\right) ^{-1}=\allowbreak \left( 
\begin{array}{cc}
\frac x{\gamma ^2} & \alpha ^2\frac y{\gamma ^4} \\ 
\frac 1{\alpha ^2}z & \frac 1{\gamma ^2}t \\ 
\frac 1\gamma \frac u\alpha & \frac 1{\gamma ^3}v\alpha%
\end{array}
\right) . 
\]

The total number of algebras here is 
\[
2p^{4}+4p^{3}+8p^{2}+14p+11+4\gcd (p-1,3)+3\gcd (p-1,4). 
\]

\subsection{Case 17}

\[
\langle
a,b,c\,|%
\,cb,bac,caa-bab,cac-baa,pa-x_{1}baa-x_{2}bab,pb-x_{3}baa-x_{4}bab,pc-x_{5}baa-x_{6}bab\rangle . 
\]%
$L_{3}$ is generated by $baa$ and $bab$ and if we let 
\[
\left( 
\begin{array}{l}
pa \\ 
pb \\ 
pc%
\end{array}%
\right) =A\left( 
\begin{array}{l}
baa \\ 
bab%
\end{array}%
\right) 
\]%
then the isomorphism classes of algebras satisfying these commutator
relations correspond to the orbits of $3\times 2$ matrices $A$ under the
action 
\[
A\rightarrow \left( 
\begin{array}{lll}
\alpha & 0 & 0 \\ 
0 & \alpha ^{-1}\gamma ^{2} & 0 \\ 
0 & 0 & \gamma%
\end{array}%
\right) A\left( 
\begin{array}{ll}
\alpha \gamma ^{2} & 0 \\ 
0 & \alpha ^{2}\gamma%
\end{array}%
\right) ^{-1} 
\]%
or 
\[
A\rightarrow \left( 
\begin{array}{lll}
\alpha & 0 & 0 \\ 
0 & 0 & \alpha ^{-1}\gamma ^{2} \\ 
0 & \gamma & 0%
\end{array}%
\right) A\left( 
\begin{array}{ll}
0 & \alpha \gamma ^{2} \\ 
\alpha ^{2}\gamma & 0%
\end{array}%
\right) ^{-1} 
\]%
with $\alpha ^{3}=\gamma ^{3}$. 
\[
\left( 
\begin{array}{lll}
\alpha & 0 & 0 \\ 
0 & \alpha ^{-1}\gamma ^{2} & 0 \\ 
0 & 0 & \gamma%
\end{array}%
\right) \left( 
\begin{array}{ll}
x & y \\ 
z & t \\ 
u & v%
\end{array}%
\right) \left( 
\begin{array}{ll}
\alpha \gamma ^{2} & 0 \\ 
0 & \alpha ^{2}\gamma%
\end{array}%
\right) ^{-1}=\allowbreak \left( 
\begin{array}{cc}
\frac{x}{\gamma ^{2}} & \frac{1}{\alpha }\frac{y}{\gamma } \\ 
\frac{1}{\alpha ^{2}}z & \frac{1}{\alpha ^{3}}\gamma t \\ 
\frac{1}{\gamma }\frac{u}{\alpha } & \frac{v}{\alpha ^{2}}%
\end{array}%
\right) 
\]%
\[
\left( 
\begin{array}{lll}
\alpha & 0 & 0 \\ 
0 & 0 & \alpha ^{-1}\gamma ^{2} \\ 
0 & \gamma & 0%
\end{array}%
\right) \left( 
\begin{array}{ll}
x & y \\ 
z & t \\ 
u & v%
\end{array}%
\right) \left( 
\begin{array}{ll}
0 & \alpha \gamma ^{2} \\ 
\alpha ^{2}\gamma & 0%
\end{array}%
\right) ^{-1}=\allowbreak \left( 
\begin{array}{cc}
\frac{y}{\gamma ^{2}} & \frac{1}{\alpha }\frac{x}{\gamma } \\ 
\frac{v}{\alpha ^{2}} & \frac{1}{\alpha ^{3}}\gamma u \\ 
\frac{1}{\gamma }\frac{t}{\alpha } & \frac{1}{\alpha ^{2}}z%
\end{array}%
\right) 
\]

If $p\neq 1\func{mod}3$ then $\alpha =\gamma $ and the number of orbits is 
\[
p^5+p^4+p^3+p^2+p+2+(p^2+p+1)\gcd (p-1,4)/2. 
\]

If $p=1\func{mod}3$ then $\alpha =\gamma $ or $\xi \gamma $ or $\xi ^2\gamma 
$ where $\xi ^3=1$. The number of orbits is then

\[
(p^5+p^4+p^3+p^2+7p+10)/3+(p^2+p+1)\gcd (p-1,4)/2 
\]

So in general the number of orbits is 
\[
\allowbreak (p^4+2p^3+3p^2+4p+2)\frac{p-1}{\gcd (p-1,3)}+3p+4+(p^2+p+1)\gcd
(p-1,4)/2 
\]

\subsection{Case 18}

\[
\langle a,b,c\,|\,cb,bac,caa-\omega
bab,cac-baa,pa-x_{1}baa-x_{2}bab,pb-x_{3}baa-x_{4}bab,pc-x_{5}baa-x_{6}bab%
\rangle \;(p=1\func{mod}3). 
\]%
This case is very similar to Case 17, though we do not have as many
automorphisms. $L_{3}$ is generated by $baa$ and $bab$ and if we let 
\[
\left( 
\begin{array}{l}
pa \\ 
pb \\ 
pc%
\end{array}%
\right) =A\left( 
\begin{array}{l}
baa \\ 
bab%
\end{array}%
\right) 
\]%
then the isomorphism classes of algebras satisfying these commutator
relations correspond to the orbits of $3\times 2$ matrices $A$ under the
action 
\[
A\rightarrow \left( 
\begin{array}{lll}
\alpha & 0 & 0 \\ 
0 & \alpha ^{-1}\gamma ^{2} & 0 \\ 
0 & 0 & \gamma%
\end{array}%
\right) A\left( 
\begin{array}{ll}
\alpha \gamma ^{2} & 0 \\ 
0 & \alpha ^{2}\gamma%
\end{array}%
\right) ^{-1} 
\]%
with $\alpha ^{3}=\gamma ^{3}$. 
\[
\left( 
\begin{array}{lll}
\alpha & 0 & 0 \\ 
0 & \alpha ^{-1}\gamma ^{2} & 0 \\ 
0 & 0 & \gamma%
\end{array}%
\right) \left( 
\begin{array}{ll}
x & y \\ 
z & t \\ 
u & v%
\end{array}%
\right) \left( 
\begin{array}{ll}
\alpha \gamma ^{2} & 0 \\ 
0 & \alpha ^{2}\gamma%
\end{array}%
\right) ^{-1}=\allowbreak \left( 
\begin{array}{cc}
\frac{x}{\gamma ^{2}} & \frac{1}{\alpha }\frac{y}{\gamma } \\ 
\frac{1}{\alpha ^{2}}z & \frac{1}{\alpha ^{3}}\gamma t \\ 
\frac{1}{\gamma }\frac{u}{\alpha } & \frac{v}{\alpha ^{2}}%
\end{array}%
\right) 
\]

The number of algebras is 
\[
(2p^5+2p^4+2p^3+2p^2+14p+17)/3 
\]

Combining Case 17 and Case 18, the total number of algebras in the two cases
is 
\[
\allowbreak p^5+p^4+p^3+p^2-2p-\frac 32+(3p+\frac 72)\gcd
(p-1,3)+(p^2+p+1)\gcd (p-1,4)/2 
\]

\subsection{Case 19}

\[
\langle
a,b,c\,|%
\,cb,baa,caa,cac,pa-x_{1}bab-x_{2}bac,pb-x_{3}bab-x_{4}bac,pc-x_{5}bab-x_{6}bac\rangle . 
\]%
$L_{3}$ is generated by $bab$ and $bac$ and if we let 
\[
\left( 
\begin{array}{l}
pa \\ 
pb \\ 
pc%
\end{array}%
\right) =A\left( 
\begin{array}{l}
bab \\ 
bac%
\end{array}%
\right) 
\]%
then the isomorphism classes of algebras satisfying these commutator
relations correspond to the orbits of $3\times 2$ matrices $A$ under the
action 
\[
A\rightarrow \left( 
\begin{array}{lll}
\alpha & 0 & 0 \\ 
0 & \beta & \gamma \\ 
0 & 0 & \delta%
\end{array}%
\right) A\left( 
\begin{array}{ll}
\alpha \beta ^{2} & 2\alpha \beta \gamma \\ 
0 & \alpha \beta \delta%
\end{array}%
\right) ^{-1} 
\]

\[
\left( 
\begin{array}{lll}
\alpha & 0 & 0 \\ 
0 & \beta & \gamma \\ 
0 & 0 & \delta%
\end{array}%
\right) \left( 
\begin{array}{ll}
x & y \\ 
z & t \\ 
u & v%
\end{array}%
\right) \left( 
\begin{array}{ll}
\alpha \beta ^{2} & 2\alpha \beta \gamma \\ 
0 & \alpha \beta \delta%
\end{array}%
\right) ^{-1} 
\]
\[
=\left( 
\begin{array}{cc}
\frac{x}{\beta ^{2}} & \frac{y}{\beta \delta }-2\frac{x}{\beta ^{2}}\frac{%
\gamma }{\delta } \\ 
\frac{1}{\alpha \beta ^{2}}\left( u\gamma +z\beta \right) & \frac{1}{\alpha
\beta \delta }\left( t\beta +v\gamma \right) -\frac{2}{\alpha \beta ^{2}}%
\frac{\gamma }{\delta }\left( u\gamma +z\beta \right) \\ 
\frac{u}{\alpha \beta ^{2}}\delta & \frac{v}{\alpha \beta }-2\frac{u}{\alpha
\beta ^{2}}\gamma%
\end{array}%
\right) . 
\]%
$\allowbreak $

The total number of algebras in this case is $2p^{2}+11p+27+\gcd (p-1,4)$.

\subsection{Case 20}

\[
\langle
a,b,c\,|%
\,cb,baa,caa-bab,cac,pa-x_{1}bab-x_{2}bac,pb-x_{3}bab-x_{4}bac,pc-x_{5}bab-x_{6}bac\rangle . 
\]%
$L_{3}$ is generated by $bab$ and $bac$ and if we let 
\[
\left( 
\begin{array}{l}
pa \\ 
pb \\ 
pc%
\end{array}%
\right) =A\left( 
\begin{array}{l}
bab \\ 
bac%
\end{array}%
\right) 
\]%
then the isomorphism classes of algebras satisfying these commutator
relations correspond to the orbits of $3\times 2$ matrices $A$ under the
action 
\[
A\rightarrow \left( 
\begin{array}{lll}
\alpha & 0 & 0 \\ 
0 & \beta & 0 \\ 
0 & 0 & \alpha ^{-1}\beta ^{2}%
\end{array}%
\right) A\left( 
\begin{array}{ll}
\alpha \beta ^{2} & 0 \\ 
0 & \beta ^{3}%
\end{array}%
\right) ^{-1} 
\]%
\[
\left( 
\begin{array}{lll}
\alpha & 0 & 0 \\ 
0 & \beta & 0 \\ 
0 & 0 & \alpha ^{-1}\beta ^{2}%
\end{array}%
\right) \left( 
\begin{array}{ll}
x & y \\ 
z & t \\ 
u & v%
\end{array}%
\right) \left( 
\begin{array}{ll}
\alpha \beta ^{2} & 0 \\ 
0 & \beta ^{3}%
\end{array}%
\right) ^{-1}=\allowbreak \left( 
\begin{array}{cc}
\frac{x}{\beta ^{2}} & \alpha \frac{y}{\beta ^{3}} \\ 
\frac{1}{\beta }\frac{z}{\alpha } & \frac{1}{\beta ^{2}}t \\ 
\frac{1}{\alpha ^{2}}u & \frac{1}{\alpha \beta }v%
\end{array}%
\right) . 
\]

The total number of algebras here is 
\[
2p^{4}+4p^{3}+6p^{2}+11p+11+2\gcd (p-1,3)+(p+1)\gcd (p-1,4). 
\]

\subsection{Case 21}

\[
\langle
a,b,c\,|%
\,cb,bab-baa,caa,cac,pa-x_{1}baa-x_{2}bac,pb-x_{3}baa-x_{4}bac,pc-x_{5}baa-x_{6}bac\rangle . 
\]%
$L_{3}$ is generated by $baa$ and $bac$ and if we let 
\[
\left( 
\begin{array}{l}
pa \\ 
pb \\ 
pc%
\end{array}%
\right) =A\left( 
\begin{array}{l}
baa \\ 
bac%
\end{array}%
\right) 
\]%
then the isomorphism classes of algebras satisfying these commutator
relations correspond to the orbits of $3\times 2$ matrices $A$ under the
action 
\[
A\rightarrow \left( 
\begin{array}{lll}
\alpha & 0 & 2\beta \\ 
0 & \alpha & \beta \\ 
0 & 0 & \gamma%
\end{array}%
\right) A\left( 
\begin{array}{ll}
\alpha ^{3} & 2\alpha ^{2}\beta \\ 
0 & \alpha ^{2}\gamma%
\end{array}%
\right) ^{-1}. 
\]

\[
\left( 
\begin{array}{lll}
\alpha & 0 & 2\beta \\ 
0 & \alpha & \beta \\ 
0 & 0 & \gamma%
\end{array}%
\right) \left( 
\begin{array}{ll}
x & y \\ 
z & t \\ 
u & v%
\end{array}%
\right) \left( 
\begin{array}{ll}
\alpha ^{3} & 2\alpha ^{2}\beta \\ 
0 & \alpha ^{2}\gamma%
\end{array}%
\right) ^{-1} 
\]
\[
=\left( 
\begin{array}{cc}
\frac{1}{\alpha ^{3}}\left( 2u\beta +x\alpha \right) & \frac{1}{\alpha
^{2}\gamma }\left( 2v\beta +y\alpha \right) -\frac{2}{\alpha ^{3}}\frac{%
\beta }{\gamma }\left( 2u\beta +x\alpha \right) \\ 
\frac{1}{\alpha ^{3}}\left( u\beta +z\alpha \right) & \frac{1}{\alpha
^{2}\gamma }\left( t\alpha +v\beta \right) -\frac{2}{\alpha ^{3}}\frac{\beta 
}{\gamma }\left( u\beta +z\alpha \right) \\ 
\frac{u}{\alpha ^{3}}\gamma & \frac{v}{\alpha ^{2}}-2\frac{u}{\alpha ^{3}}%
\beta%
\end{array}%
\right) . 
\]%
$\allowbreak $

The total number of algebras in this case is 
\[
2p^{3}+6p^{2}+7p+7+(p+1)\gcd (p-1,4). 
\]

\subsection{Case 22}

\[
\langle a,b,c\,|\,cb,baa,caa,cac-\omega
bab,pa-x_{1}bab-x_{2}bac,pb-x_{3}bab-x_{4}bac,pc-x_{5}bab-x_{6}bac\rangle . 
\]%
$L_{3}$ is generated by $bab$ and $bac$ and if we let 
\[
\left( 
\begin{array}{l}
pa \\ 
pb \\ 
pc%
\end{array}%
\right) =A\left( 
\begin{array}{l}
bab \\ 
bac%
\end{array}%
\right) 
\]%
then the isomorphism classes of algebras satisfying these commutator
relations correspond to the orbits of $3\times 2$ matrices $A$ under the
action 
\[
A\rightarrow \left( 
\begin{array}{lll}
\alpha & 0 & 0 \\ 
0 & \omega \beta & \pm \gamma \\ 
0 & \omega \gamma & \pm \omega \beta%
\end{array}%
\right) A\left( 
\begin{array}{ll}
\omega \alpha (\omega \beta ^{2}+\gamma ^{2}) & \pm 2\omega \alpha \beta
\gamma \\ 
2\omega ^{2}\alpha \beta \gamma & \pm \omega \alpha (\omega \beta
^{2}+\gamma ^{2})%
\end{array}%
\right) ^{-1}. 
\]

The total number of algebras in Case 22 is 
\[
(2p^{3}+3p^{2}+3p+13-\gcd (p-1,3)+(p+1)\gcd (p-1,4))/2. 
\]

\subsection{Case 23}

\[
\langle a,b,c\,|\,cb,baa,caa-bac,cac-\omega
bab,pa-x_{1}bab-x_{2}bac,pb-x_{3}bab-x_{4}bac,pc-x_{5}bab-x_{6}bac\rangle . 
\]%
$L_{3}$ is generated by $bab$ and $bac$ and if we let 
\[
\left( 
\begin{array}{l}
pa \\ 
pb \\ 
pc%
\end{array}%
\right) =A\left( 
\begin{array}{l}
bab \\ 
bac%
\end{array}%
\right) 
\]%
then the isomorphism classes of algebras satisfying these commutator
relations correspond to the orbits of $3\times 2$ matrices $A$ under the
action 
\[
A\rightarrow \left( 
\begin{array}{lll}
\alpha & 0 & 0 \\ 
0 & \alpha & 0 \\ 
0 & 0 & \pm \alpha%
\end{array}%
\right) A\left( 
\begin{array}{ll}
\alpha ^{3} & 0 \\ 
0 & \pm \alpha ^{3}%
\end{array}%
\right) ^{-1} 
\]%
or when $p=2\func{mod}3$ and $12\omega \beta ^{2}=-1$, 
\[
A\rightarrow \left( 
\begin{array}{lll}
4\omega \alpha \beta & -3\omega \alpha \beta & \frac{\alpha }{2} \\ 
0 & -2\omega \alpha \beta & \alpha \\ 
0 & \pm \omega \alpha & \mp 2\omega \alpha \beta%
\end{array}%
\right) A\left( 
\begin{array}{ll}
\frac{8}{3}\omega ^{2}\alpha ^{3}\beta & \frac{4}{3}\omega \alpha ^{3} \\ 
\pm \frac{4}{3}\omega ^{2}\alpha ^{3} & \pm \frac{8}{3}\omega ^{2}\alpha
^{3}\beta%
\end{array}%
\right) ^{-1}. 
\]

Now 
\[
\left( 
\begin{array}{lll}
\alpha & 0 & 0 \\ 
0 & \alpha & 0 \\ 
0 & 0 & \pm \alpha%
\end{array}
\right) \left( 
\begin{array}{ll}
x & y \\ 
z & t \\ 
u & v%
\end{array}
\right) \left( 
\begin{array}{ll}
\alpha ^3 & 0 \\ 
0 & \pm \alpha ^3%
\end{array}
\right) ^{-1}=\allowbreak \left( 
\begin{array}{cc}
\frac 1{\alpha ^2}x & \pm \frac 1{\alpha ^2}y \\ 
\frac 1{\alpha ^2}z & \pm \frac 1{\alpha ^2}t \\ 
\pm \frac 1{\alpha ^2}u & \frac 1{\alpha ^2}v%
\end{array}
\right) 
\]
and so if $p=1\func{mod}3$ there are $p^5+p^4+p^3+p^2+p+2+(p^2+p+1)\gcd
(p-1,4)/2$ algebras.

When $p=2\func{mod}3$ the number of algebras here is 
\[
\allowbreak \frac{1}{3}p^{5}+\frac{1}{3}p^{4}+\frac{1}{3}p^{3}+\frac{1}{3}%
p^{2}+p+2+(p^{2}+p+1)\gcd (p-1,4)/2. 
\]

\subsection{Case 24}

\[
\langle a,b,c\,|\,cb,baa,caa-kbab-bac,cac-\omega
bab,pa-x_{1}bab-x_{2}bac,pb-x_{3}bab-x_{4}bac,pc-x_{5}bab-x_{6}bac\rangle
\;(p=2\func{mod}3). 
\]%
where $k$ is any (fixed) integer which is not a value of 
\[
\frac{\lambda (\lambda ^{2}+3\omega \mu ^{2})}{\mu (3\lambda ^{2}+\omega \mu
^{2})}\func{mod}p. 
\]%
$L_{3}$ is generated by $bab$ and $bac$ and if we let 
\[
\left( 
\begin{array}{l}
pa \\ 
pb \\ 
pc%
\end{array}%
\right) =A\left( 
\begin{array}{l}
bab \\ 
bac%
\end{array}%
\right) 
\]%
then the isomorphism classes of algebras satisfying these commutator
relations correspond to the orbits of $3\times 2$ matrices $A$ under the
action 
\[
A\rightarrow \left( 
\begin{array}{lll}
\alpha & 0 & 0 \\ 
0 & \alpha & 0 \\ 
0 & 0 & \alpha%
\end{array}%
\right) A\left( 
\begin{array}{ll}
\alpha ^{3} & 0 \\ 
0 & \alpha ^{3}%
\end{array}%
\right) ^{-1} 
\]%
and 
\[
A\rightarrow \left( 
\begin{array}{lll}
-4\alpha & k\alpha \beta +3\alpha & 3k\omega ^{-1}\alpha +\alpha \beta \\ 
0 & 2\alpha & 2\alpha \beta \\ 
0 & 2\omega \alpha \beta & 2\alpha%
\end{array}%
\right) A\left( 
\begin{array}{ll}
32\alpha ^{3} & -32\alpha ^{3}\beta \\ 
-32\omega \alpha ^{3}\beta & 32\alpha ^{3}%
\end{array}%
\right) ^{-1} 
\]%
where $\omega \beta ^{2}=-3$.

The number of orbits is 
\[
\allowbreak \frac{2}{3}p^{5}+\frac{2}{3}p^{4}+\frac{2}{3}p^{3}+\frac{2}{3}%
p^{2}+2p+3. 
\]

The total number of algebras from Case 23 and Case 24 is 
\[
p^{5}+p^{4}+p^{3}+p^{2}+4p+\frac{13}{2}-(p+\frac{3}{2})\gcd
(p-1,3)+(p^{2}+p+1)\gcd (p-1,4)/2. 
\]

The total number of algebras from cases 17, 18, 23 and 24 is 
\[
p^5+p^4+p^3+p^2+2p+5+(2p+2)\gcd (p-1,3)+(p^2+p+1)\gcd (p-1,4). 
\]

\end{document}
