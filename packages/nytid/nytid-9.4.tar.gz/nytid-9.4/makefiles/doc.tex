\section{Introduction and usage}% ===> this file was generated automatically by noweave --- better not edit it

When working with large sets of documents we sometimes want to do some 
operations on them, \eg print them or convert them between formats.
This include file provides exactly that.

We provide a target {\Tt{}print\nwendquote} which prints its prerequisites, see 
\cref{Printing} for details.
It prints them using {\Tt{}{\$}{\nwlbrace}LPR{\nwrbrace}\ {\$}{\nwlbrace}LPRFLAGS{\nwrbrace}\ <file>\nwendquote}.
Each of these variables has specialized {\Tt{}LPR-<file>\nwendquote} and {\Tt{}LPRFLAGS-<file>\nwendquote} 
versions, which means that different documents can be printed with different 
commands and flags.

We also provide a target {\Tt{}wc\nwendquote} which counts the words of its prerequisites 
(\cref{WordCounting}).
The counting is done with {\Tt{}{\$}{\nwlbrace}WC{\nwrbrace}\ {\$}{\nwlbrace}WCFLAGS{\nwrbrace}\ <file>\nwendquote}.
There is also a {\Tt{}todo\nwendquote} target (\cref{ToDoLists}) which grep's its 
prerequisites for common todo comments (\eg TODO, XXX, FIXME).

Each of {\Tt{}wc\nwendquote} and {\Tt{}todo\nwendquote} has the specialized {\Tt{}-<file>\nwendquote} versions of the 
variables, same as for {\Tt{}print\nwendquote}.
However, {\Tt{}wc\nwendquote} and {\Tt{}todo\nwendquote} additionally have a {\Tt{}PREWC\nwendquote} and {\Tt{}PRETODO\nwendquote} 
commands which pipe their output to the {\Tt{}WC\nwendquote} and {\Tt{}TODO\nwendquote} commands, 
respectively.
These variables have {\Tt{}-<file>\nwendquote} versions, but no flags variable.

Finally, we also provide a set of suffix rules for automatic conversion between 
different formats, see \cref{FormatConversion} for details about the formats.
In general, we provide the variable {\Tt{}CONVERT.a.b\nwendquote} which can be used in 
recipes to convert a file of format {\Tt{}.a\nwendquote} to a file of format {\Tt{}.b\nwendquote}.
The command run is designed to fit a rule of the form {\Tt{}{\%}.b:\ {\%}.a\nwendquote}.


\section{Implementation}

Since the makefile is designed for inclusion, we want to ensure that it is not 
included more than once --- like we do in C and C++.
Then first comes our variables described above followed by the targets.
\nwfilename{doc.mk.nw}\nwbegincode{1}\sublabel{NW24Dyoa-30nQvD-1}\nwmargintag{{\nwtagstyle{}\subpageref{NW24Dyoa-30nQvD-1}}}\moddef{doc.mk~{\nwtagstyle{}\subpageref{NW24Dyoa-30nQvD-1}}}\endmoddef\nwstartdeflinemarkup\nwenddeflinemarkup
ifndef DOC_MK
DOC_MK=true

INCLUDE_MAKEFILES?=.
include $\{INCLUDE_MAKEFILES\}/portability.mk

\LA{}variables~{\nwtagstyle{}\subpageref{NW24Dyoa-29jjeg-1}}\RA{}
\LA{}target for printing~{\nwtagstyle{}\subpageref{NW24Dyoa-2hdAVM-1}}\RA{}
\LA{}target for word counting~{\nwtagstyle{}\subpageref{NW24Dyoa-1mgWVy-1}}\RA{}
\LA{}target for to-do lists~{\nwtagstyle{}\subpageref{NW24Dyoa-oXGPJ-1}}\RA{}
\LA{}suffix rules for format conversion~{\nwtagstyle{}\subpageref{NW24Dyoa-3n2nVJ-1}}\RA{}

endif
\nwnotused{doc.mk}\nwendcode{}\nwbegindocs{2}\nwdocspar

\subsection{Printing}
\label{Printing}

We provide a target {\Tt{}print\nwendquote} to print all documents in a set.
The usage is simply that documents are added as prerequisites, then the target 
prints all documents in its dependency list.
The printing is done using the {\Tt{}lpr\nwendquote} command by default.
However, this can be changed with the {\Tt{}LPR\nwendquote} variable.
\nwenddocs{}\nwbegincode{3}\sublabel{NW24Dyoa-29jjeg-1}\nwmargintag{{\nwtagstyle{}\subpageref{NW24Dyoa-29jjeg-1}}}\moddef{variables~{\nwtagstyle{}\subpageref{NW24Dyoa-29jjeg-1}}}\endmoddef\nwstartdeflinemarkup\nwusesondefline{\\{NW24Dyoa-30nQvD-1}}\nwprevnextdefs{\relax}{NW24Dyoa-29jjeg-2}\nwenddeflinemarkup
LPR?=       lpr
LPRFLAGS?=
\nwalsodefined{\\{NW24Dyoa-29jjeg-2}\\{NW24Dyoa-29jjeg-3}\\{NW24Dyoa-29jjeg-4}\\{NW24Dyoa-29jjeg-5}\\{NW24Dyoa-29jjeg-6}\\{NW24Dyoa-29jjeg-7}\\{NW24Dyoa-29jjeg-8}\\{NW24Dyoa-29jjeg-9}\\{NW24Dyoa-29jjeg-A}\\{NW24Dyoa-29jjeg-B}\\{NW24Dyoa-29jjeg-C}}\nwused{\\{NW24Dyoa-30nQvD-1}}\nwendcode{}\nwbegindocs{4}\nwdocspar

If {\Tt{}lpr\nwendquote} is used, we note that the files added as prerequisites for the 
{\Tt{}print\nwendquote} target must be printable by {\Tt{}lpr\nwendquote}, \eg we must supply 
PostScript-files instead of PDF-files.
Fortunately, the automatic file format conversion (\cref{FormatConversion}) 
solves most of those problems.
For example, if you want to print a PDF-file {\Tt{}something.pdf\nwendquote}, then just add 
{\Tt{}something.ps\nwendquote} as a prerequisite to print and the suffix rules below will do 
the rest.

The implementation is quite simple.
We will iterate through the list of prerequisites and process them one by
one.
For each document we will check if there is an overriding setting for the 
printing command and its arguments, if there is not we use the default set 
above.
\nwenddocs{}\nwbegincode{5}\sublabel{NW24Dyoa-2hdAVM-1}\nwmargintag{{\nwtagstyle{}\subpageref{NW24Dyoa-2hdAVM-1}}}\moddef{target for printing~{\nwtagstyle{}\subpageref{NW24Dyoa-2hdAVM-1}}}\endmoddef\nwstartdeflinemarkup\nwusesondefline{\\{NW24Dyoa-30nQvD-1}}\nwenddeflinemarkup
.PHONY: print
print:
  $(foreach doc,$^,\\
    $(if $\{LPR-$\{doc\}\},$\{LPR-$\{doc\}\},$\{LPR\}) \\
    $(if $\{LPRFLAGS-$\{doc\}\},$\{LPRFLAGS-$\{doc\}\},$\{LPRFLAGS\}) \\
    $\{doc\};)
\nwused{\\{NW24Dyoa-30nQvD-1}}\nwendcode{}\nwbegindocs{6}\nwdocspar

\subsection{Counting words}
\label{WordCounting}

We provide a {\Tt{}wc\nwendquote} target which counts the words in its prerequisites.
The files added as prerequisites must thus be text files.
Similarly as for print, there are suffix rules to convert \eg TeX-files to 
plain text files using {\Tt{}detex\nwendquote}.

The implementation is similar to that for {\Tt{}print\nwendquote}.
The counting is done using the {\Tt{}wc\nwendquote} command by default, but we allow 
overrides using the following variable.
\nwenddocs{}\nwbegincode{7}\sublabel{NW24Dyoa-29jjeg-2}\nwmargintag{{\nwtagstyle{}\subpageref{NW24Dyoa-29jjeg-2}}}\moddef{variables~{\nwtagstyle{}\subpageref{NW24Dyoa-29jjeg-1}}}\plusendmoddef\nwstartdeflinemarkup\nwusesondefline{\\{NW24Dyoa-30nQvD-1}}\nwprevnextdefs{NW24Dyoa-29jjeg-1}{NW24Dyoa-29jjeg-3}\nwenddeflinemarkup
WC?=        wc
WCFLAGS?=   -w
\nwused{\\{NW24Dyoa-30nQvD-1}}\nwendcode{}\nwbegindocs{8}\nwdocspar

We will simply iterate through the list of prerequisites and process them one 
by one using {\Tt{}wc\nwendquote}.
We first print the name followed by a colon, then we print the word count.
Similarly as above, we check for each document whether there is an overriding 
setting for the word counting command.
We also check if there is a preprocessing command, \eg it might be useful to 
run detex(1) on TeX files before counting the words.
\nwenddocs{}\nwbegincode{9}\sublabel{NW24Dyoa-1mgWVy-1}\nwmargintag{{\nwtagstyle{}\subpageref{NW24Dyoa-1mgWVy-1}}}\moddef{target for word counting~{\nwtagstyle{}\subpageref{NW24Dyoa-1mgWVy-1}}}\endmoddef\nwstartdeflinemarkup\nwusesondefline{\\{NW24Dyoa-30nQvD-1}}\nwenddeflinemarkup
.PHONY: wc
wc:
  $(foreach doc,$^,echo -n "$\{doc\}: "; $\{CAT\} $\{doc\} | \\
    $(if $\{PREWC-$\{doc\}\},$\{PREWC-$\{doc\}\} |,$(if $\{PREWC\},$\{PREWC\} |,)) \\
    $(if $\{WC-$\{doc\}\},$\{WC-$\{doc\}\},$\{WC\}) \\
    $(if $\{WCFLAGS-$\{doc\}\},$\{WCFLAGS-$\{doc\}\},$\{WCFLAGS\});)
\nwused{\\{NW24Dyoa-30nQvD-1}}\nwendcode{}\nwbegindocs{10}\nwdocspar

\subsection{To-do lists}%
\label{ToDoLists}

Similarly to the {\Tt{}wc\nwendquote} target, we would also like to add a {\Tt{}todo\nwendquote} target 
which generates a to-do list from the to-do comments in the source files (\ie 
\enquote{TODO}, \enquote{XXX} or \enquote{FIXME}).
\nwenddocs{}\nwbegincode{11}\sublabel{NW24Dyoa-oXGPJ-1}\nwmargintag{{\nwtagstyle{}\subpageref{NW24Dyoa-oXGPJ-1}}}\moddef{target for to-do lists~{\nwtagstyle{}\subpageref{NW24Dyoa-oXGPJ-1}}}\endmoddef\nwstartdeflinemarkup\nwusesondefline{\\{NW24Dyoa-30nQvD-1}}\nwenddeflinemarkup
.PHONY: todo
todo:
  $(foreach doc,$^,echo "$\{doc\}: "; $\{CAT\} $\{doc\} | \\
    $(if $\{PRETODO-$\{doc\}\},$\{PRETODO-$\{doc\}\} |,$(if $\{PRETODO\},$\{PRETODO\} |,)) \\
    $(if $\{TODO-$\{doc\}\},$\{TODO-$\{doc\}\},$\{TODO\}) \\
    $(if $\{TODOFLAGS-$\{doc\}\},$\{TODOFLAGS-$\{doc\}\},$\{TODOFLAGS\});echo;)
\nwused{\\{NW24Dyoa-30nQvD-1}}\nwendcode{}\nwbegindocs{12}\nwdocspar

We will use the grep(1) utility to grep for these flags in the files.
\nwenddocs{}\nwbegincode{13}\sublabel{NW24Dyoa-29jjeg-3}\nwmargintag{{\nwtagstyle{}\subpageref{NW24Dyoa-29jjeg-3}}}\moddef{variables~{\nwtagstyle{}\subpageref{NW24Dyoa-29jjeg-1}}}\plusendmoddef\nwstartdeflinemarkup\nwusesondefline{\\{NW24Dyoa-30nQvD-1}}\nwprevnextdefs{NW24Dyoa-29jjeg-2}{NW24Dyoa-29jjeg-4}\nwenddeflinemarkup
TODO?=        $\{GREP\} "\\(XXX\\|TODO\\|FIXME\\)"
TODOFLAGS?=
\nwused{\\{NW24Dyoa-30nQvD-1}}\nwendcode{}\nwbegindocs{14}\nwdocspar

\subsection{Format conversion}
\label{FormatConversion}

We often want to convert a file from one format to another.
We will now write a pattern along the lines of make's conventions for 
compilation targets~\cite[Sect.~10.2, second to last paragraph]{GNUMake}.
\Ie, {\Tt{}CONVERT.a.b\nwendquote} will convert a file with suffix {\Tt{}.a\nwendquote} to a format with 
suffix {\Tt{}.b\nwendquote}.
However, we will only provide pattern rules for the most common direction, \eg 
from {\Tt{}.a\nwendquote} to {\Tt{}.b\nwendquote} but not from {\Tt{}.b\nwendquote} to {\Tt{}.a\nwendquote}.
Having both will cause make to drop one due to circular dependencies.

We provide the following patterns:
\nwenddocs{}\nwbegincode{15}\sublabel{NW24Dyoa-3n2nVJ-1}\nwmargintag{{\nwtagstyle{}\subpageref{NW24Dyoa-3n2nVJ-1}}}\moddef{suffix rules for format conversion~{\nwtagstyle{}\subpageref{NW24Dyoa-3n2nVJ-1}}}\endmoddef\nwstartdeflinemarkup\nwusesondefline{\\{NW24Dyoa-30nQvD-1}}\nwenddeflinemarkup
\LA{}PS to PDF~{\nwtagstyle{}\subpageref{NW24Dyoa-EtapP-1}}\RA{}
\LA{}DVI to PS~{\nwtagstyle{}\subpageref{NW24Dyoa-1cjAZS-1}}\RA{}
\LA{}ODF to PDF~{\nwtagstyle{}\subpageref{NW24Dyoa-3UjpCj-1}}\RA{}

\LA{}SVG to PDF~{\nwtagstyle{}\subpageref{NW24Dyoa-2DBQKE-1}}\RA{}
\LA{}SVG to PS~{\nwtagstyle{}\subpageref{NW24Dyoa-3pi6CK-1}}\RA{}
\LA{}DIA to TeX~{\nwtagstyle{}\subpageref{NW24Dyoa-4eYbec-1}}\RA{}
\LA{}XCF to PNG~{\nwtagstyle{}\subpageref{NW24Dyoa-3Ti4ja-1}}\RA{}

\LA{}MD to TeX~{\nwtagstyle{}\subpageref{NW24Dyoa-t0DBM-1}}\RA{}
\LA{}MD, TeX to HTML~{\nwtagstyle{}\subpageref{NW24Dyoa-4SnFzz-1}}\RA{}
\LA{}TeX to text~{\nwtagstyle{}\subpageref{NW24Dyoa-24otFo-1}}\RA{}
\nwused{\\{NW24Dyoa-30nQvD-1}}\nwendcode{}\nwbegindocs{16}\nwdocspar

\subsubsection{Document formats}

To convert PDFs to PostScript format, we use the {\Tt{}pdf2ps\nwendquote} command by default.
\nwenddocs{}\nwbegincode{17}\sublabel{NW24Dyoa-29jjeg-4}\nwmargintag{{\nwtagstyle{}\subpageref{NW24Dyoa-29jjeg-4}}}\moddef{variables~{\nwtagstyle{}\subpageref{NW24Dyoa-29jjeg-1}}}\plusendmoddef\nwstartdeflinemarkup\nwusesondefline{\\{NW24Dyoa-30nQvD-1}}\nwprevnextdefs{NW24Dyoa-29jjeg-3}{NW24Dyoa-29jjeg-5}\nwenddeflinemarkup
PDF2PS?=          pdf2ps
PDF2PSFLAGS?=
CONVERT.pdf.ps?=  $\{PDF2PS\} $\{PDF2PSFLAGS\} $<

PS2PDF?=          ps2pdf
PS2PDFFLAGS?=
CONVERT.ps.pdf?=  $\{PS2PDF\} $\{PS2PDFFLAGS\} $<
\nwused{\\{NW24Dyoa-30nQvD-1}}\nwendcode{}\nwbegindocs{18}This allows us to specify the rules as follows.
\nwenddocs{}\nwbegincode{19}\sublabel{NW24Dyoa-EtapP-1}\nwmargintag{{\nwtagstyle{}\subpageref{NW24Dyoa-EtapP-1}}}\moddef{PS to PDF~{\nwtagstyle{}\subpageref{NW24Dyoa-EtapP-1}}}\endmoddef\nwstartdeflinemarkup\nwusesondefline{\\{NW24Dyoa-3n2nVJ-1}}\nwenddeflinemarkup
%.pdf: %.ps
  $\{CONVERT.ps.pdf\}
\nwused{\\{NW24Dyoa-3n2nVJ-1}}\nwendcode{}\nwbegindocs{20}\nwdocspar

We do similarly for DVI-files that we want to convert to PostScript.
\nwenddocs{}\nwbegincode{21}\sublabel{NW24Dyoa-29jjeg-5}\nwmargintag{{\nwtagstyle{}\subpageref{NW24Dyoa-29jjeg-5}}}\moddef{variables~{\nwtagstyle{}\subpageref{NW24Dyoa-29jjeg-1}}}\plusendmoddef\nwstartdeflinemarkup\nwusesondefline{\\{NW24Dyoa-30nQvD-1}}\nwprevnextdefs{NW24Dyoa-29jjeg-4}{NW24Dyoa-29jjeg-6}\nwenddeflinemarkup
DVIPS?=           dvips
DVIPSFLAGS?=
CONVERT.dvi.ps?=  $\{DVIPS\} $\{DVIPSFLAGS\} $<
\nwused{\\{NW24Dyoa-30nQvD-1}}\nwendcode{}\nwbegindocs{22}With those variables we let
\nwenddocs{}\nwbegincode{23}\sublabel{NW24Dyoa-1cjAZS-1}\nwmargintag{{\nwtagstyle{}\subpageref{NW24Dyoa-1cjAZS-1}}}\moddef{DVI to PS~{\nwtagstyle{}\subpageref{NW24Dyoa-1cjAZS-1}}}\endmoddef\nwstartdeflinemarkup\nwusesondefline{\\{NW24Dyoa-3n2nVJ-1}}\nwenddeflinemarkup
%.ps: %.dvi
  $\{CONVERT.dvi.ps\}
\nwused{\\{NW24Dyoa-3n2nVJ-1}}\nwendcode{}\nwbegindocs{24}\nwdocspar

There is no good conversion program for the Open Document Format (ODF) files.
We will use LibreOffice although that will incur some manual labour.
\nwenddocs{}\nwbegincode{25}\sublabel{NW24Dyoa-29jjeg-6}\nwmargintag{{\nwtagstyle{}\subpageref{NW24Dyoa-29jjeg-6}}}\moddef{variables~{\nwtagstyle{}\subpageref{NW24Dyoa-29jjeg-1}}}\plusendmoddef\nwstartdeflinemarkup\nwusesondefline{\\{NW24Dyoa-30nQvD-1}}\nwprevnextdefs{NW24Dyoa-29jjeg-5}{NW24Dyoa-29jjeg-7}\nwenddeflinemarkup
ODF2PDF?=         soffice --convert-to pdf
ODF2PDFFLAGS?=    --headless
CONVERT.odt.pdf?= $\{ODF2PDF\} $\{ODF2PDFFLAGS\} $<
CONVERT.ods.pdf?= $\{ODF2PDF\} $\{ODF2PDFFLAGS\} $<
CONVERT.odg.pdf?= $\{ODF2PDF\} $\{ODF2PDFFLAGS\} $<
CONVERT.odp.pdf?= $\{ODF2PDF\} $\{ODF2PDFFLAGS\} $<
CONVERT.doc.pdf?= $\{ODF2PDF\} $\{ODF2PDFFLAGS\} $<
CONVERT.docx.pdf?=$\{ODF2PDF\} $\{ODF2PDFFLAGS\} $<
CONVERT.xls.pdf?= $\{ODF2PDF\} $\{ODF2PDFFLAGS\} $<
CONVERT.xlsx.pdf?=$\{ODF2PDF\} $\{ODF2PDFFLAGS\} $<
CONVERT.ppt.pdf?= $\{ODF2PDF\} $\{ODF2PDFFLAGS\} $<
CONVERT.pptx.pdf?=$\{ODF2PDF\} $\{ODF2PDFFLAGS\} $<
\nwused{\\{NW24Dyoa-30nQvD-1}}\nwendcode{}\nwbegindocs{26}Then we can generate the suffix rules as follows.
\nwenddocs{}\nwbegincode{27}\sublabel{NW24Dyoa-3UjpCj-1}\nwmargintag{{\nwtagstyle{}\subpageref{NW24Dyoa-3UjpCj-1}}}\moddef{ODF to PDF~{\nwtagstyle{}\subpageref{NW24Dyoa-3UjpCj-1}}}\endmoddef\nwstartdeflinemarkup\nwusesondefline{\\{NW24Dyoa-3n2nVJ-1}}\nwenddeflinemarkup
define def_convert_rule
%.pdf: %.$(1)
  $\{CONVERT.$(1).pdf\}
endef
$(foreach suf,odt ods odg odp doc docx xls xlsx ppt pptx,\\
  $(eval $(call def_convert_rule,$\{suf\})))
\nwused{\\{NW24Dyoa-3n2nVJ-1}}\nwendcode{}\nwbegindocs{28}\nwdocspar

\subsubsection{Figure formats}

Usually we want to keep figures in their source form, so that we can still edit
them later.
However, just as usually, we cannot use the source form directly in TeX 
documents, so we want to convert them to TeX or PDF.

When working with SVG-files, there are two things: the graphics and the text in 
the graphics.
We will use Inkscape for working with SVGs, because Inkscape allows us to 
export the graphics part as PDF and all text in it as TeX.
Unlike previously, we will only allow flags for {\Tt{}inkscape\nwendquote} to be set.
\nwenddocs{}\nwbegincode{29}\sublabel{NW24Dyoa-29jjeg-7}\nwmargintag{{\nwtagstyle{}\subpageref{NW24Dyoa-29jjeg-7}}}\moddef{variables~{\nwtagstyle{}\subpageref{NW24Dyoa-29jjeg-1}}}\plusendmoddef\nwstartdeflinemarkup\nwusesondefline{\\{NW24Dyoa-30nQvD-1}}\nwprevnextdefs{NW24Dyoa-29jjeg-6}{NW24Dyoa-29jjeg-8}\nwenddeflinemarkup
INKSCAPE?=        inkscape
INKSCAPEFLAGS?=   -D -z --export-latex
CONVERT.svg.pdf?= $\{INKSCAPE\} $\{INKSCAPEFLAGS\} --file=$< --export-pdf=$@
CONVERT.svg.ps?=  $\{INKSCAPE\} $\{INKSCAPEFLAGS\} --file=$< --export-ps=$@
CONVERT.svg.eps?= $\{INKSCAPE\} $\{INKSCAPEFLAGS\} --file=$< --export-eps=$@
\nwused{\\{NW24Dyoa-30nQvD-1}}\nwendcode{}\nwbegindocs{30}\nwdocspar
\nwenddocs{}\nwbegincode{31}\sublabel{NW24Dyoa-2DBQKE-1}\nwmargintag{{\nwtagstyle{}\subpageref{NW24Dyoa-2DBQKE-1}}}\moddef{SVG to PDF~{\nwtagstyle{}\subpageref{NW24Dyoa-2DBQKE-1}}}\endmoddef\nwstartdeflinemarkup\nwusesondefline{\\{NW24Dyoa-3n2nVJ-1}}\nwenddeflinemarkup
%.pdf: %.svg
  $\{CONVERT.svg.pdf\}
\nwused{\\{NW24Dyoa-3n2nVJ-1}}\nwendcode{}\nwbegindocs{32}We can thus create similar rules for the formats PS and EPS, instead of PDF.
\nwenddocs{}\nwbegincode{33}\sublabel{NW24Dyoa-3pi6CK-1}\nwmargintag{{\nwtagstyle{}\subpageref{NW24Dyoa-3pi6CK-1}}}\moddef{SVG to PS~{\nwtagstyle{}\subpageref{NW24Dyoa-3pi6CK-1}}}\endmoddef\nwstartdeflinemarkup\nwusesondefline{\\{NW24Dyoa-3n2nVJ-1}}\nwenddeflinemarkup
%.ps %.eps: %.svg
  $\{CONVERT.svg$(suffix $@)\}
\nwused{\\{NW24Dyoa-3n2nVJ-1}}\nwendcode{}\nwbegindocs{34}\nwdocspar

Dia is a useful tool for making figures over network topologies etc.
Fortunately, Dia can output native TeX.
Similarly to Inkscape, we will only provide flags for Dia.
\nwenddocs{}\nwbegincode{35}\sublabel{NW24Dyoa-29jjeg-8}\nwmargintag{{\nwtagstyle{}\subpageref{NW24Dyoa-29jjeg-8}}}\moddef{variables~{\nwtagstyle{}\subpageref{NW24Dyoa-29jjeg-1}}}\plusendmoddef\nwstartdeflinemarkup\nwusesondefline{\\{NW24Dyoa-30nQvD-1}}\nwprevnextdefs{NW24Dyoa-29jjeg-7}{NW24Dyoa-29jjeg-9}\nwenddeflinemarkup
DIA?=             dia
DIAFLAGS?=
CONVERT.dia.tex?= $\{DIA\} $\{DIAFLAGS\} -e $@ -t pgf-tex $<
\nwused{\\{NW24Dyoa-30nQvD-1}}\nwendcode{}\nwbegindocs{36}That gives the suffix rule as follows.
\nwenddocs{}\nwbegincode{37}\sublabel{NW24Dyoa-4eYbec-1}\nwmargintag{{\nwtagstyle{}\subpageref{NW24Dyoa-4eYbec-1}}}\moddef{DIA to TeX~{\nwtagstyle{}\subpageref{NW24Dyoa-4eYbec-1}}}\endmoddef\nwstartdeflinemarkup\nwusesondefline{\\{NW24Dyoa-3n2nVJ-1}}\nwenddeflinemarkup
%.tex: %.dia
  $\{CONVERT.dia.tex\}
\nwused{\\{NW24Dyoa-3n2nVJ-1}}\nwendcode{}\nwbegindocs{38}\nwdocspar

Gimp is a useful tool for raster graphics.
We can convert Gimps own format XCF to PNG using the ImageMagic library.
\nwenddocs{}\nwbegincode{39}\sublabel{NW24Dyoa-29jjeg-9}\nwmargintag{{\nwtagstyle{}\subpageref{NW24Dyoa-29jjeg-9}}}\moddef{variables~{\nwtagstyle{}\subpageref{NW24Dyoa-29jjeg-1}}}\plusendmoddef\nwstartdeflinemarkup\nwusesondefline{\\{NW24Dyoa-30nQvD-1}}\nwprevnextdefs{NW24Dyoa-29jjeg-8}{NW24Dyoa-29jjeg-A}\nwenddeflinemarkup
XCF2PNGFLAGS?=    -flatten
XCF2PNG?=         convert $\{XCF2PNGFLAGS\} $< $@
TRIM?=            convert -trim $@ $@
CONVERT.xcf.png?= $\{XCF2PNG\} && $\{TRIM\}
\nwused{\\{NW24Dyoa-30nQvD-1}}\nwendcode{}\nwbegindocs{40}Then we can have the following rule:
\nwenddocs{}\nwbegincode{41}\sublabel{NW24Dyoa-3Ti4ja-1}\nwmargintag{{\nwtagstyle{}\subpageref{NW24Dyoa-3Ti4ja-1}}}\moddef{XCF to PNG~{\nwtagstyle{}\subpageref{NW24Dyoa-3Ti4ja-1}}}\endmoddef\nwstartdeflinemarkup\nwusesondefline{\\{NW24Dyoa-3n2nVJ-1}}\nwenddeflinemarkup
%.png: %.xcf
  $\{CONVERT.xcf.png\}
\nwused{\\{NW24Dyoa-3n2nVJ-1}}\nwendcode{}\nwbegindocs{42}\nwdocspar

\subsubsection{Text-based formats}

The conversion of the text-based formats differ from the formats above.
Most of these tools automatically write their output to stdout, which is 
customary when working with text in the terminal.

We use the {\Tt{}pandoc\nwendquote} program to convert between Markdown and TeX.
\nwenddocs{}\nwbegincode{43}\sublabel{NW24Dyoa-29jjeg-A}\nwmargintag{{\nwtagstyle{}\subpageref{NW24Dyoa-29jjeg-A}}}\moddef{variables~{\nwtagstyle{}\subpageref{NW24Dyoa-29jjeg-1}}}\plusendmoddef\nwstartdeflinemarkup\nwusesondefline{\\{NW24Dyoa-30nQvD-1}}\nwprevnextdefs{NW24Dyoa-29jjeg-9}{NW24Dyoa-29jjeg-B}\nwenddeflinemarkup
MD2TEX?=        pandoc
MD2TEXFLAGS?=   -s
CONVERT.md.tex?=$\{MD2TEX\} $\{MD2TEXFLAGS\} -o $@ $<

TEX2MD?=        pandoc
TEX2MDFLAGS?=   -s
CONVERT.tex.md?=$\{TEX2MD\} $\{TEX2MDFLAGS\} -o $@ $<
\nwused{\\{NW24Dyoa-30nQvD-1}}\nwendcode{}\nwbegindocs{44}This gives the following suffix rules.
\nwenddocs{}\nwbegincode{45}\sublabel{NW24Dyoa-t0DBM-1}\nwmargintag{{\nwtagstyle{}\subpageref{NW24Dyoa-t0DBM-1}}}\moddef{MD to TeX~{\nwtagstyle{}\subpageref{NW24Dyoa-t0DBM-1}}}\endmoddef\nwstartdeflinemarkup\nwusesondefline{\\{NW24Dyoa-3n2nVJ-1}}\nwenddeflinemarkup
%.md: %.tex
  $\{CONVERT.tex.md\}
\nwused{\\{NW24Dyoa-3n2nVJ-1}}\nwendcode{}\nwbegindocs{46}\nwdocspar

Sometimes we want to convert to HTML, we will use {\Tt{}pandoc\nwendquote} to convert both 
Markdown and LaTeX to HTML.
\nwenddocs{}\nwbegincode{47}\sublabel{NW24Dyoa-29jjeg-B}\nwmargintag{{\nwtagstyle{}\subpageref{NW24Dyoa-29jjeg-B}}}\moddef{variables~{\nwtagstyle{}\subpageref{NW24Dyoa-29jjeg-1}}}\plusendmoddef\nwstartdeflinemarkup\nwusesondefline{\\{NW24Dyoa-30nQvD-1}}\nwprevnextdefs{NW24Dyoa-29jjeg-A}{NW24Dyoa-29jjeg-C}\nwenddeflinemarkup
MD2HTML?=           pandoc
MD2HTMLFLAGS?=      -s
CONVERT.md.html?=   $\{MD2HTML\} $\{MD2HTMLFLAGS\} -o $@ $<

TEX2HTML?=          pandoc
TEX2HTMLFLAGS?=     -s
CONVERT.tex.html?=  $\{TEX2HTML\} $\{TEX2HTMLFLAGS\} -o $@ $<
\nwused{\\{NW24Dyoa-30nQvD-1}}\nwendcode{}\nwbegindocs{48}Then we can provide following two rules:
\nwenddocs{}\nwbegincode{49}\sublabel{NW24Dyoa-4SnFzz-1}\nwmargintag{{\nwtagstyle{}\subpageref{NW24Dyoa-4SnFzz-1}}}\moddef{MD, TeX to HTML~{\nwtagstyle{}\subpageref{NW24Dyoa-4SnFzz-1}}}\endmoddef\nwstartdeflinemarkup\nwusesondefline{\\{NW24Dyoa-3n2nVJ-1}}\nwenddeflinemarkup
define to_html_rule
%.html: %.$(1)
  $\{CONVERT.$(1).html\}
endef
$(foreach suf,md tex,$(eval $(call to_html_rule,$\{suf\})))
\nwused{\\{NW24Dyoa-3n2nVJ-1}}\nwendcode{}\nwbegindocs{50}\nwdocspar

There are times when we want to convert out TeX-files to plain text, \eg to 
count the words.
To do this we simply use the {\Tt{}detex\nwendquote} program.
\nwenddocs{}\nwbegincode{51}\sublabel{NW24Dyoa-29jjeg-C}\nwmargintag{{\nwtagstyle{}\subpageref{NW24Dyoa-29jjeg-C}}}\moddef{variables~{\nwtagstyle{}\subpageref{NW24Dyoa-29jjeg-1}}}\plusendmoddef\nwstartdeflinemarkup\nwusesondefline{\\{NW24Dyoa-30nQvD-1}}\nwprevnextdefs{NW24Dyoa-29jjeg-B}{\relax}\nwenddeflinemarkup
TEX2TEXT?=        detex
TEX2TEXTFLAGS?=
CONVERT.tex.txt?= $\{TEX2TEXT\} $\{TEX2TEXTFLAGS\} -o $@ $<
\nwused{\\{NW24Dyoa-30nQvD-1}}\nwendcode{}\nwbegindocs{52}This gives us the following suffix rule.
\nwenddocs{}\nwbegincode{53}\sublabel{NW24Dyoa-24otFo-1}\nwmargintag{{\nwtagstyle{}\subpageref{NW24Dyoa-24otFo-1}}}\moddef{TeX to text~{\nwtagstyle{}\subpageref{NW24Dyoa-24otFo-1}}}\endmoddef\nwstartdeflinemarkup\nwusesondefline{\\{NW24Dyoa-3n2nVJ-1}}\nwenddeflinemarkup
%.txt: %.tex
  $\{CONVERT.tex.txt\}
\nwused{\\{NW24Dyoa-3n2nVJ-1}}\nwendcode{}

\nwixlogsorted{c}{{DIA to TeX}{NW24Dyoa-4eYbec-1}{\nwixu{NW24Dyoa-3n2nVJ-1}\nwixd{NW24Dyoa-4eYbec-1}}}%
\nwixlogsorted{c}{{doc.mk}{NW24Dyoa-30nQvD-1}{\nwixd{NW24Dyoa-30nQvD-1}}}%
\nwixlogsorted{c}{{DVI to PS}{NW24Dyoa-1cjAZS-1}{\nwixu{NW24Dyoa-3n2nVJ-1}\nwixd{NW24Dyoa-1cjAZS-1}}}%
\nwixlogsorted{c}{{MD to TeX}{NW24Dyoa-t0DBM-1}{\nwixu{NW24Dyoa-3n2nVJ-1}\nwixd{NW24Dyoa-t0DBM-1}}}%
\nwixlogsorted{c}{{MD, TeX to HTML}{NW24Dyoa-4SnFzz-1}{\nwixu{NW24Dyoa-3n2nVJ-1}\nwixd{NW24Dyoa-4SnFzz-1}}}%
\nwixlogsorted{c}{{ODF to PDF}{NW24Dyoa-3UjpCj-1}{\nwixu{NW24Dyoa-3n2nVJ-1}\nwixd{NW24Dyoa-3UjpCj-1}}}%
\nwixlogsorted{c}{{PS to PDF}{NW24Dyoa-EtapP-1}{\nwixu{NW24Dyoa-3n2nVJ-1}\nwixd{NW24Dyoa-EtapP-1}}}%
\nwixlogsorted{c}{{suffix rules for format conversion}{NW24Dyoa-3n2nVJ-1}{\nwixu{NW24Dyoa-30nQvD-1}\nwixd{NW24Dyoa-3n2nVJ-1}}}%
\nwixlogsorted{c}{{SVG to PDF}{NW24Dyoa-2DBQKE-1}{\nwixu{NW24Dyoa-3n2nVJ-1}\nwixd{NW24Dyoa-2DBQKE-1}}}%
\nwixlogsorted{c}{{SVG to PS}{NW24Dyoa-3pi6CK-1}{\nwixu{NW24Dyoa-3n2nVJ-1}\nwixd{NW24Dyoa-3pi6CK-1}}}%
\nwixlogsorted{c}{{target for printing}{NW24Dyoa-2hdAVM-1}{\nwixu{NW24Dyoa-30nQvD-1}\nwixd{NW24Dyoa-2hdAVM-1}}}%
\nwixlogsorted{c}{{target for to-do lists}{NW24Dyoa-oXGPJ-1}{\nwixu{NW24Dyoa-30nQvD-1}\nwixd{NW24Dyoa-oXGPJ-1}}}%
\nwixlogsorted{c}{{target for word counting}{NW24Dyoa-1mgWVy-1}{\nwixu{NW24Dyoa-30nQvD-1}\nwixd{NW24Dyoa-1mgWVy-1}}}%
\nwixlogsorted{c}{{TeX to text}{NW24Dyoa-24otFo-1}{\nwixu{NW24Dyoa-3n2nVJ-1}\nwixd{NW24Dyoa-24otFo-1}}}%
\nwixlogsorted{c}{{variables}{NW24Dyoa-29jjeg-1}{\nwixu{NW24Dyoa-30nQvD-1}\nwixd{NW24Dyoa-29jjeg-1}\nwixd{NW24Dyoa-29jjeg-2}\nwixd{NW24Dyoa-29jjeg-3}\nwixd{NW24Dyoa-29jjeg-4}\nwixd{NW24Dyoa-29jjeg-5}\nwixd{NW24Dyoa-29jjeg-6}\nwixd{NW24Dyoa-29jjeg-7}\nwixd{NW24Dyoa-29jjeg-8}\nwixd{NW24Dyoa-29jjeg-9}\nwixd{NW24Dyoa-29jjeg-A}\nwixd{NW24Dyoa-29jjeg-B}\nwixd{NW24Dyoa-29jjeg-C}}}%
\nwixlogsorted{c}{{XCF to PNG}{NW24Dyoa-3Ti4ja-1}{\nwixu{NW24Dyoa-3n2nVJ-1}\nwixd{NW24Dyoa-3Ti4ja-1}}}%
\nwbegindocs{54}\nwdocspar

\nwenddocs{}
