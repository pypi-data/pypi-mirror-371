\label{portability.mk}% ===> this file was generated automatically by noweave --- better not edit it
\section{Introduction}

The purpose of this include file is to improve portability of the include 
files.
The make(1) utility itself already provides certain portability between 
platforms, here we want to extend this portability.
I.e.\ we provide variables which substitute to system-specific commands which 
corresponds to the expected action.
For instance, MacOS uses an ancient version of {\Tt{}unzip\nwendquote}, a version which does 
not support the option {\Tt{}-DD\nwendquote} which is desirable.
So, on MacOS the variable {\Tt{}UNZIP\nwendquote} will substitute to {\Tt{}unzip\nwendquote} which on other
systems it will substitute to {\Tt{}unzip\ -DD\nwendquote}.
Another examples is BSD-systems, which does not use the GNU versions of {\Tt{}sed\nwendquote}
and {\Tt{}grep\nwendquote} (and {\Tt{}make\nwendquote}).
On these systems {\Tt{}SED\nwendquote} will substitute to {\Tt{}gsed\nwendquote}, which is the GNU version 
of the command.
Probably the reader can skip this chapter on a first reading.

The include file is structures similarly to a header file in C.
We use the same technique to prevent multiple inclusions.
The outline is as follows.
\nwfilename{portability.mk.nw}\nwbegincode{1}\sublabel{NW1AE0du-okr1C-1}\nwmargintag{{\nwtagstyle{}\subpageref{NW1AE0du-okr1C-1}}}\moddef{portability.mk~{\nwtagstyle{}\subpageref{NW1AE0du-okr1C-1}}}\endmoddef\nwstartdeflinemarkup\nwenddeflinemarkup
ifndef PORTABILITY_MK
PORTABILITY_MK=true

\LA{}system-specific configuration~{\nwtagstyle{}\subpageref{NW1AE0du-45XUZU-1}}\RA{}

\LA{}standard unix commands~{\nwtagstyle{}\subpageref{NW1AE0du-1VdKMJ-1}}\RA{}
\LA{}networking commands~{\nwtagstyle{}\subpageref{NW1AE0du-26HyRR-1}}\RA{}
\LA{}compressed files and archives~{\nwtagstyle{}\subpageref{NW1AE0du-3TkHp6-1}}\RA{}

endif
\nwnotused{portability.mk}\nwendcode{}\nwbegindocs{2}Since this file provides system-dependent configuration, we allow the user to
provide a system-wide configuration file.
\nwenddocs{}\nwbegincode{3}\sublabel{NW1AE0du-45XUZU-1}\nwmargintag{{\nwtagstyle{}\subpageref{NW1AE0du-45XUZU-1}}}\moddef{system-specific configuration~{\nwtagstyle{}\subpageref{NW1AE0du-45XUZU-1}}}\endmoddef\nwstartdeflinemarkup\nwusesondefline{\\{NW1AE0du-okr1C-1}}\nwenddeflinemarkup
PORTABILITY_CONF?=  $\{HOME\}/.mk.conf /etc/mk.conf
-include $\{PORTABILITY_CONF\}
\nwused{\\{NW1AE0du-okr1C-1}}\nwendcode{}\nwbegindocs{4}The file in {\Tt{}/etc/mk.conf\nwendquote} is commonly available in BSDs.
However, since these files might not exist, make(1) should not yield a fatal 
error if the include directive fails.


\section{Standard Unix commands}

In this section we provide default commands with options for the standard Unix 
command line.
More specifically, we cover the following areas.
\nwenddocs{}\nwbegincode{5}\sublabel{NW1AE0du-1VdKMJ-1}\nwmargintag{{\nwtagstyle{}\subpageref{NW1AE0du-1VdKMJ-1}}}\moddef{standard unix commands~{\nwtagstyle{}\subpageref{NW1AE0du-1VdKMJ-1}}}\endmoddef\nwstartdeflinemarkup\nwusesondefline{\\{NW1AE0du-okr1C-1}}\nwenddeflinemarkup
\LA{}file system commands~{\nwtagstyle{}\subpageref{NW1AE0du-2hTQq2-1}}\RA{}
\LA{}printing file contents~{\nwtagstyle{}\subpageref{NW1AE0du-1QBbQP-1}}\RA{}
\LA{}opening files depending on file type~{\nwtagstyle{}\subpageref{NW1AE0du-1j2O2M-1}}\RA{}
\LA{}filtering and transforming file contents~{\nwtagstyle{}\subpageref{NW1AE0du-2RSbVo-1}}\RA{}
\LA{}statistics on file contents~{\nwtagstyle{}\subpageref{NW1AE0du-1Dsqad-1}}\RA{}
\nwused{\\{NW1AE0du-okr1C-1}}\nwendcode{}\nwbegindocs{6}\nwdocspar

\subsection{File system commands}

We commonly use the commands to interact with the file system.
The following basic commands cover most uses.
\nwenddocs{}\nwbegincode{7}\sublabel{NW1AE0du-2hTQq2-1}\nwmargintag{{\nwtagstyle{}\subpageref{NW1AE0du-2hTQq2-1}}}\moddef{file system commands~{\nwtagstyle{}\subpageref{NW1AE0du-2hTQq2-1}}}\endmoddef\nwstartdeflinemarkup\nwusesondefline{\\{NW1AE0du-1VdKMJ-1}}\nwenddeflinemarkup
MV?=      mv
CP?=      cp -R
LN?=      ln -sf
MKDIR?=   mkdir -p
MKTMPDIR?=mktemp -d
CHOWN?=   chown -R
CHMOD?=   chmod -R
\nwused{\\{NW1AE0du-1VdKMJ-1}}\nwendcode{}\nwbegindocs{8}The make(1) utility already sets {\Tt{}RM\ =\ rm\ -f\nwendquote} by default~\cite[Sect.\ 
10.3]{GNUMake}, so we need not repeat it.

\subsection{Viewing file contents}

Quite commonly we want to open files with the user's desired application, \eg 
to open PDFs in the user's PDF reader.
For this we use the xdg-open(1) utility.
\nwenddocs{}\nwbegincode{9}\sublabel{NW1AE0du-1j2O2M-1}\nwmargintag{{\nwtagstyle{}\subpageref{NW1AE0du-1j2O2M-1}}}\moddef{opening files depending on file type~{\nwtagstyle{}\subpageref{NW1AE0du-1j2O2M-1}}}\endmoddef\nwstartdeflinemarkup\nwusesondefline{\\{NW1AE0du-1VdKMJ-1}}\nwenddeflinemarkup
OPEN?=    xdg-open
\nwused{\\{NW1AE0du-1VdKMJ-1}}\nwendcode{}\nwbegindocs{10}However, for text files, we prefer to just print the contents to standard 
output.
\nwenddocs{}\nwbegincode{11}\sublabel{NW1AE0du-1QBbQP-1}\nwmargintag{{\nwtagstyle{}\subpageref{NW1AE0du-1QBbQP-1}}}\moddef{printing file contents~{\nwtagstyle{}\subpageref{NW1AE0du-1QBbQP-1}}}\endmoddef\nwstartdeflinemarkup\nwusesondefline{\\{NW1AE0du-1VdKMJ-1}}\nwenddeflinemarkup
CAT?=     cat
\nwused{\\{NW1AE0du-1VdKMJ-1}}\nwendcode{}\nwbegindocs{12}\nwdocspar

\subsection{Filtering and transformations}

Two of the most frequently used utilities are sed(1) and grep(1).
The version of these that we want to use is the GNU version.
On Linux systems, this is the default.
On BSDs, however, they are available prefixed with the letter \enquote{g}.
The same goes to the make(1) utility, which means that we can use that fact to 
check for this.
\nwenddocs{}\nwbegincode{13}\sublabel{NW1AE0du-2RSbVo-1}\nwmargintag{{\nwtagstyle{}\subpageref{NW1AE0du-2RSbVo-1}}}\moddef{filtering and transforming file contents~{\nwtagstyle{}\subpageref{NW1AE0du-2RSbVo-1}}}\endmoddef\nwstartdeflinemarkup\nwusesondefline{\\{NW1AE0du-1VdKMJ-1}}\nwprevnextdefs{\relax}{NW1AE0du-2RSbVo-2}\nwenddeflinemarkup
ifeq ($\{MAKE\},gmake)
SED?=     gsed
SEDex?=   gsed -E
else
SED?=     sed
SEDex?=   sed -E
endif
\nwalsodefined{\\{NW1AE0du-2RSbVo-2}}\nwused{\\{NW1AE0du-1VdKMJ-1}}\nwendcode{}\nwbegindocs{14}Similarly, we let
\nwenddocs{}\nwbegincode{15}\sublabel{NW1AE0du-2RSbVo-2}\nwmargintag{{\nwtagstyle{}\subpageref{NW1AE0du-2RSbVo-2}}}\moddef{filtering and transforming file contents~{\nwtagstyle{}\subpageref{NW1AE0du-2RSbVo-1}}}\plusendmoddef\nwstartdeflinemarkup\nwusesondefline{\\{NW1AE0du-1VdKMJ-1}}\nwprevnextdefs{NW1AE0du-2RSbVo-1}{\relax}\nwenddeflinemarkup
ifeq ($\{MAKE\},gmake)
GREP=     ggrep
GREPex=   ggrep -E
else
GREP=     grep
GREPex=   grep -E
endif
\nwused{\\{NW1AE0du-1VdKMJ-1}}\nwendcode{}\nwbegindocs{16}\nwdocspar

\subsection{Statistics}

We also need to count words in a few places.
We use wc(1) for this.
\nwenddocs{}\nwbegincode{17}\sublabel{NW1AE0du-1Dsqad-1}\nwmargintag{{\nwtagstyle{}\subpageref{NW1AE0du-1Dsqad-1}}}\moddef{statistics on file contents~{\nwtagstyle{}\subpageref{NW1AE0du-1Dsqad-1}}}\endmoddef\nwstartdeflinemarkup\nwusesondefline{\\{NW1AE0du-1VdKMJ-1}}\nwenddeflinemarkup
WC?=      wc
WCw?=     wc -w
\nwused{\\{NW1AE0du-1VdKMJ-1}}\nwendcode{}\nwbegindocs{18}\nwdocspar


\section{Networking commands}

We also need some network related commands.
\nwenddocs{}\nwbegincode{19}\sublabel{NW1AE0du-26HyRR-1}\nwmargintag{{\nwtagstyle{}\subpageref{NW1AE0du-26HyRR-1}}}\moddef{networking commands~{\nwtagstyle{}\subpageref{NW1AE0du-26HyRR-1}}}\endmoddef\nwstartdeflinemarkup\nwusesondefline{\\{NW1AE0du-okr1C-1}}\nwenddeflinemarkup
\LA{}fetching files~{\nwtagstyle{}\subpageref{NW1AE0du-rjStN-1}}\RA{}
\LA{}remote execution~{\nwtagstyle{}\subpageref{NW1AE0du-2gLZR6-1}}\RA{}
\nwused{\\{NW1AE0du-okr1C-1}}\nwendcode{}\nwbegindocs{20}\nwdocspar

We have some common commands for fetching and copying files between remote 
hosts.
\nwenddocs{}\nwbegincode{21}\sublabel{NW1AE0du-rjStN-1}\nwmargintag{{\nwtagstyle{}\subpageref{NW1AE0du-rjStN-1}}}\moddef{fetching files~{\nwtagstyle{}\subpageref{NW1AE0du-rjStN-1}}}\endmoddef\nwstartdeflinemarkup\nwusesondefline{\\{NW1AE0du-26HyRR-1}}\nwenddeflinemarkup
CURL?=    curl
SFTP?=    sftp
SCP?=     scp -r
\nwused{\\{NW1AE0du-26HyRR-1}}\nwendcode{}\nwbegindocs{22}\nwdocspar

We also need commands for remote execution.
\nwenddocs{}\nwbegincode{23}\sublabel{NW1AE0du-2gLZR6-1}\nwmargintag{{\nwtagstyle{}\subpageref{NW1AE0du-2gLZR6-1}}}\moddef{remote execution~{\nwtagstyle{}\subpageref{NW1AE0du-2gLZR6-1}}}\endmoddef\nwstartdeflinemarkup\nwusesondefline{\\{NW1AE0du-26HyRR-1}}\nwenddeflinemarkup
SSH?=     ssh
\nwused{\\{NW1AE0du-26HyRR-1}}\nwendcode{}\nwbegindocs{24}\nwdocspar


\section{Compressed files and archives}

We want to provide functionality to make it easy to uncompress files or extract 
files from archives of different kinds.
We will construct two functionalities.
\nwenddocs{}\nwbegincode{25}\sublabel{NW1AE0du-3TkHp6-1}\nwmargintag{{\nwtagstyle{}\subpageref{NW1AE0du-3TkHp6-1}}}\moddef{compressed files and archives~{\nwtagstyle{}\subpageref{NW1AE0du-3TkHp6-1}}}\endmoddef\nwstartdeflinemarkup\nwusesondefline{\\{NW1AE0du-okr1C-1}}\nwenddeflinemarkup
\LA{}variables for compression programs~{\nwtagstyle{}\subpageref{NW1AE0du-8WF6G-1}}\RA{}

\LA{}variables for archive programs~{\nwtagstyle{}\subpageref{NW1AE0du-5Fqx6-1}}\RA{}
\LA{}general pattern rule for archiving~{\nwtagstyle{}\subpageref{NW1AE0du-4HTPNs-1}}\RA{}
\LA{}function to generate extraction targets~{\nwtagstyle{}\subpageref{NW1AE0du-3ARMxi-1}}\RA{}
\nwused{\\{NW1AE0du-okr1C-1}}\nwendcode{}\nwbegindocs{26}Both will use the type of construction outlined in~\cite[Sect.\ 
10.2]{GNUMake}:
the variable {\Tt{}EXTRACT.suf\nwendquote} ({\Tt{}UNCOMPRESS.suf\nwendquote}) will contain the command to 
extract a file from an archive (decompress a file) with suffix {\Tt{}.suf\nwendquote};
the variable {\Tt{}ARCHIVE.suf\nwendquote} ({\Tt{}COMPRESS.suf\nwendquote}) will contain the command to 
update an archive of suffix {\Tt{}.suf\nwendquote} with a file (compress a file).

\subsection{Compressing and uncompressing files}
\label{CompressingFiles}

A compressed file is a file whose data is compressed --- this is not 
necessarily an archive.
A compressed file can be uncompressed, \ie the compression is removed.
Compressed files usually get the added suffix of the compression algorithm, 
\eg a {\Tt{}.tar\nwendquote} file usually get the suffix {\Tt{}.tar.gz\nwendquote} when it is also 
compressed using gzip(1).
Another common file to compress is PostScript, \ie turning {\Tt{}.ps\nwendquote} to 
{\Tt{}.ps.gz\nwendquote}.
We want to form pattern rules for the compression and uncompression operations.

There are, of course, a myriad different compression formats.
We will let the variable {\Tt{}COMPRESS{\_}SUFFIXES\nwendquote} and {\Tt{}UNCOMPRESS{\_}SUFFIXES\nwendquote} 
contain space-separated lists of suffixes supported for the two operations.

To compress a file, we simply passes its contents through a compression 
program, \eg gzip(1) (gets the {\Tt{}.gz\nwendquote} suffix).
We can use the following general pattern rule for compression and then use 
{\Tt{}COMPRESS{\_}SUFFIXES\nwendquote} to automatically generate all the pattern 
rules\footnote{
  Note that the pattern rules in the code blocks
  {\Tt{}\LA{}general pattern rule for compression~{\nwtagstyle{}\subpageref{NW1AE0du-1eeSjh-1}}\RA{}\nwendquote} and
  {\Tt{}\LA{}general pattern rule for uncompression~{\nwtagstyle{}\subpageref{NW1AE0du-1ce4LT-1}}\RA{}\nwendquote} are not included in
  {\Tt{}\LA{}compressed files and archives~{\nwtagstyle{}\subpageref{NW1AE0du-3TkHp6-1}}\RA{}\nwendquote} above, and thus not enabled by default.
  This is due to causing circular dependencies.
}.
\nwenddocs{}\nwbegincode{27}\sublabel{NW1AE0du-1eeSjh-1}\nwmargintag{{\nwtagstyle{}\subpageref{NW1AE0du-1eeSjh-1}}}\moddef{general pattern rule for compression~{\nwtagstyle{}\subpageref{NW1AE0du-1eeSjh-1}}}\endmoddef\nwstartdeflinemarkup\nwenddeflinemarkup
define compress
%$(1): %
  $$\{COMPRESS$(1)\}
endef
$(foreach suf,$\{COMPRESS_SUFFIXES\},$(eval $(call compress,$\{suf\})))
\nwnotused{general pattern rule for compression}\nwendcode{}\nwbegindocs{28}In a similar fashion, we can use the following general pattern rule for 
uncompression.
\nwenddocs{}\nwbegincode{29}\sublabel{NW1AE0du-1ce4LT-1}\nwmargintag{{\nwtagstyle{}\subpageref{NW1AE0du-1ce4LT-1}}}\moddef{general pattern rule for uncompression~{\nwtagstyle{}\subpageref{NW1AE0du-1ce4LT-1}}}\endmoddef\nwstartdeflinemarkup\nwenddeflinemarkup
define uncompress
%: %$(1)
  $$\{UNCOMPRESS$(1)\}
endef
$(foreach suf,$\{UNCOMPRESS_SUFFIXES\},$(eval $(call uncompress,$\{suf\})))
\nwnotused{general pattern rule for uncompression}\nwendcode{}\nwbegindocs{30}We note that due to to the {\Tt{}call\nwendquote} and {\Tt{}eval\nwendquote} above, we must escape the 
target and prerequisite variables, {\Tt{}{\$}{\$}@\nwendquote} and {\Tt{}{\$}{\$}<\nwendquote}, respectively.

Now, let us write what we need to automatically handle the gzip(1) format.
To uncompress a gzipped file we can use gunzip(1).
\nwenddocs{}\nwbegincode{31}\sublabel{NW1AE0du-8WF6G-1}\nwmargintag{{\nwtagstyle{}\subpageref{NW1AE0du-8WF6G-1}}}\moddef{variables for compression programs~{\nwtagstyle{}\subpageref{NW1AE0du-8WF6G-1}}}\endmoddef\nwstartdeflinemarkup\nwusesondefline{\\{NW1AE0du-3TkHp6-1}}\nwprevnextdefs{\relax}{NW1AE0du-8WF6G-2}\nwenddeflinemarkup
UNCOMPRESS_SUFFIXES+= .gz .z
GUNZIP?=              gunzip
UNCOMPRESS.gz?=       $\{GUNZIP\} $<
UNCOMPRESS.z?=        $\{UNCOMPRESS.gz\}
\nwalsodefined{\\{NW1AE0du-8WF6G-2}}\nwused{\\{NW1AE0du-3TkHp6-1}}\nwendcode{}\nwbegindocs{32}To compress a file using gzip(1) we can use the following.
\nwenddocs{}\nwbegincode{33}\sublabel{NW1AE0du-8WF6G-2}\nwmargintag{{\nwtagstyle{}\subpageref{NW1AE0du-8WF6G-2}}}\moddef{variables for compression programs~{\nwtagstyle{}\subpageref{NW1AE0du-8WF6G-1}}}\plusendmoddef\nwstartdeflinemarkup\nwusesondefline{\\{NW1AE0du-3TkHp6-1}}\nwprevnextdefs{NW1AE0du-8WF6G-1}{\relax}\nwenddeflinemarkup
COMPRESS_SUFFIXES+=   .gz
GZIP?=                gzip
COMPRESS.gz?=         $\{GZIP\} $<
\nwused{\\{NW1AE0du-3TkHp6-1}}\nwendcode{}\nwbegindocs{34}\nwdocspar

\subsection{Packing and extracting from archives}
\label{portability:Archives}

We can (ab)use the archive syntax~\cite[Chap.\ 11]{GNUMake} of make(1) to 
create a pattern rule for creating archives.
This rule will work for all archive types that support adding files to an 
existing archive.
However, make(1) cannot check the modification times of these archive members, 
so they will be updated every time instead of only when necessary.

The pattern rule matches all archive member targets.
Then it determines which variable to use as recipe by looking at the suffix of 
the archive.
\nwenddocs{}\nwbegincode{35}\sublabel{NW1AE0du-4HTPNs-1}\nwmargintag{{\nwtagstyle{}\subpageref{NW1AE0du-4HTPNs-1}}}\moddef{general pattern rule for archiving~{\nwtagstyle{}\subpageref{NW1AE0du-4HTPNs-1}}}\endmoddef\nwstartdeflinemarkup\nwusesondefline{\\{NW1AE0du-3TkHp6-1}}\nwenddeflinemarkup
(%):
  $\{ARCHIVE$(suffix $@)\}
\nwused{\\{NW1AE0du-3TkHp6-1}}\nwendcode{}\nwbegindocs{36}Unlike for the compression targets above (\cref{CompressingFiles}), we do not 
need to escape {\Tt{}{\$}@\nwendquote} and
{\Tt{}{\$}<\nwendquote} --- since we have only one (lazy) evaluation.

We do not want to break the native archive functionality of make(1), so we 
provide the following to retain that.
\nwenddocs{}\nwbegincode{37}\sublabel{NW1AE0du-5Fqx6-1}\nwmargintag{{\nwtagstyle{}\subpageref{NW1AE0du-5Fqx6-1}}}\moddef{variables for archive programs~{\nwtagstyle{}\subpageref{NW1AE0du-5Fqx6-1}}}\endmoddef\nwstartdeflinemarkup\nwusesondefline{\\{NW1AE0du-3TkHp6-1}}\nwprevnextdefs{\relax}{NW1AE0du-5Fqx6-2}\nwenddeflinemarkup
ARCHIVE.a?=   ar r $@ $%
\nwalsodefined{\\{NW1AE0du-5Fqx6-2}\\{NW1AE0du-5Fqx6-3}\\{NW1AE0du-5Fqx6-4}\\{NW1AE0du-5Fqx6-5}}\nwused{\\{NW1AE0du-3TkHp6-1}}\nwendcode{}\nwbegindocs{38}\nwdocspar

Now to a more interesting format, let us create tarballs using this syntax.
We provide settings for both tar(1) and pax(1) using the tar format.
We are interested in the pax(1) command because it has an interface for regular 
expressions, \ie for filtering and transforming file names.
The BSD tar(1) has this too, but the GNU tar(1) does not.
We can use the {\Tt{}-u\nwendquote} option to both tar(1) and pax(1) to update an existing 
archive with a file.
\nwenddocs{}\nwbegincode{39}\sublabel{NW1AE0du-5Fqx6-2}\nwmargintag{{\nwtagstyle{}\subpageref{NW1AE0du-5Fqx6-2}}}\moddef{variables for archive programs~{\nwtagstyle{}\subpageref{NW1AE0du-5Fqx6-1}}}\plusendmoddef\nwstartdeflinemarkup\nwusesondefline{\\{NW1AE0du-3TkHp6-1}}\nwprevnextdefs{NW1AE0du-5Fqx6-1}{NW1AE0du-5Fqx6-3}\nwenddeflinemarkup
TAR?=         tar -u
PAX?=         pax -wzLx ustar
ARCHIVE.tar?= $\{TAR\} -f $@ $%
\nwused{\\{NW1AE0du-3TkHp6-1}}\nwendcode{}\nwbegindocs{40}We can also create zip(1) archives.
\nwenddocs{}\nwbegincode{41}\sublabel{NW1AE0du-5Fqx6-3}\nwmargintag{{\nwtagstyle{}\subpageref{NW1AE0du-5Fqx6-3}}}\moddef{variables for archive programs~{\nwtagstyle{}\subpageref{NW1AE0du-5Fqx6-1}}}\plusendmoddef\nwstartdeflinemarkup\nwusesondefline{\\{NW1AE0du-3TkHp6-1}}\nwprevnextdefs{NW1AE0du-5Fqx6-2}{NW1AE0du-5Fqx6-4}\nwenddeflinemarkup
ZIP?=         zip
ARCHIVE.zip?= $\{ZIP\} -u $@ $%
\nwused{\\{NW1AE0du-3TkHp6-1}}\nwendcode{}\nwbegindocs{42}\nwdocspar

Unfortunately, we cannot create any pattern rules for file extraction from 
archives.
However, we can provide a function which create such targets automatically.
\nwenddocs{}\nwbegincode{43}\sublabel{NW1AE0du-3ARMxi-1}\nwmargintag{{\nwtagstyle{}\subpageref{NW1AE0du-3ARMxi-1}}}\moddef{function to generate extraction targets~{\nwtagstyle{}\subpageref{NW1AE0du-3ARMxi-1}}}\endmoddef\nwstartdeflinemarkup\nwusesondefline{\\{NW1AE0du-3TkHp6-1}}\nwenddeflinemarkup
define extract
$(1): $(2)
  $$\{EXTRACT$(suffix $(2))\}
endef
\nwused{\\{NW1AE0du-3TkHp6-1}}\nwendcode{}\nwbegindocs{44}Now we only need to provide the {\Tt{}EXTRACT.XXX\nwendquote} for every type of archive we 
might want to use.
Then we can use the function {\Tt{}extract\nwendquote} in our makefiles.
Note that we are now in the same situation as for the compression targets 
(\cref{CompressingFiles}), so we must escape the variables.

We start with tarballs.
For extraction, we do not want to restore the modification times from inside 
the archive.
If we restore the modification times, then the archive will always be newer 
that the files extracted from it and thus make(1) will re-extract the file 
every time.
To prevent this we add the {\Tt{}-m\nwendquote} option to tar(1).
\nwenddocs{}\nwbegincode{45}\sublabel{NW1AE0du-5Fqx6-4}\nwmargintag{{\nwtagstyle{}\subpageref{NW1AE0du-5Fqx6-4}}}\moddef{variables for archive programs~{\nwtagstyle{}\subpageref{NW1AE0du-5Fqx6-1}}}\plusendmoddef\nwstartdeflinemarkup\nwusesondefline{\\{NW1AE0du-3TkHp6-1}}\nwprevnextdefs{NW1AE0du-5Fqx6-3}{NW1AE0du-5Fqx6-5}\nwenddeflinemarkup
UNTAR?=       tar -xm
UNPAX?=       pax -rzp m
EXTRACT.tar?= $\{UNTAR\} -f $< $@
\nwused{\\{NW1AE0du-3TkHp6-1}}\nwendcode{}\nwbegindocs{46}It will be similar for zip archives.
The option to prevent resetting the modification time for unzip(1) is {\Tt{}-DD\nwendquote}.
Unfortunately, MacOS ships with an ancient version of unzip(1), one which does 
not support the desired {\Tt{}-DD\nwendquote} option.
Hence we check if the system is Darwin, if so, we skip the {\Tt{}-DD\nwendquote} option.
\nwenddocs{}\nwbegincode{47}\sublabel{NW1AE0du-5Fqx6-5}\nwmargintag{{\nwtagstyle{}\subpageref{NW1AE0du-5Fqx6-5}}}\moddef{variables for archive programs~{\nwtagstyle{}\subpageref{NW1AE0du-5Fqx6-1}}}\plusendmoddef\nwstartdeflinemarkup\nwusesondefline{\\{NW1AE0du-3TkHp6-1}}\nwprevnextdefs{NW1AE0du-5Fqx6-4}{\relax}\nwenddeflinemarkup
ifeq ($(shell uname),Darwin)
UNZIP?=       unzip
else
UNZIP?=       unzip -DD
endif
EXTRACT.zip?= $\{UNZIP\} $< $@
\nwused{\\{NW1AE0du-3TkHp6-1}}\nwendcode{}

\nwixlogsorted{c}{{compressed files and archives}{NW1AE0du-3TkHp6-1}{\nwixu{NW1AE0du-okr1C-1}\nwixd{NW1AE0du-3TkHp6-1}}}%
\nwixlogsorted{c}{{fetching files}{NW1AE0du-rjStN-1}{\nwixu{NW1AE0du-26HyRR-1}\nwixd{NW1AE0du-rjStN-1}}}%
\nwixlogsorted{c}{{file system commands}{NW1AE0du-2hTQq2-1}{\nwixu{NW1AE0du-1VdKMJ-1}\nwixd{NW1AE0du-2hTQq2-1}}}%
\nwixlogsorted{c}{{filtering and transforming file contents}{NW1AE0du-2RSbVo-1}{\nwixu{NW1AE0du-1VdKMJ-1}\nwixd{NW1AE0du-2RSbVo-1}\nwixd{NW1AE0du-2RSbVo-2}}}%
\nwixlogsorted{c}{{function to generate extraction targets}{NW1AE0du-3ARMxi-1}{\nwixu{NW1AE0du-3TkHp6-1}\nwixd{NW1AE0du-3ARMxi-1}}}%
\nwixlogsorted{c}{{general pattern rule for archiving}{NW1AE0du-4HTPNs-1}{\nwixu{NW1AE0du-3TkHp6-1}\nwixd{NW1AE0du-4HTPNs-1}}}%
\nwixlogsorted{c}{{general pattern rule for compression}{NW1AE0du-1eeSjh-1}{\nwixd{NW1AE0du-1eeSjh-1}}}%
\nwixlogsorted{c}{{general pattern rule for uncompression}{NW1AE0du-1ce4LT-1}{\nwixd{NW1AE0du-1ce4LT-1}}}%
\nwixlogsorted{c}{{networking commands}{NW1AE0du-26HyRR-1}{\nwixu{NW1AE0du-okr1C-1}\nwixd{NW1AE0du-26HyRR-1}}}%
\nwixlogsorted{c}{{opening files depending on file type}{NW1AE0du-1j2O2M-1}{\nwixu{NW1AE0du-1VdKMJ-1}\nwixd{NW1AE0du-1j2O2M-1}}}%
\nwixlogsorted{c}{{portability.mk}{NW1AE0du-okr1C-1}{\nwixd{NW1AE0du-okr1C-1}}}%
\nwixlogsorted{c}{{printing file contents}{NW1AE0du-1QBbQP-1}{\nwixu{NW1AE0du-1VdKMJ-1}\nwixd{NW1AE0du-1QBbQP-1}}}%
\nwixlogsorted{c}{{remote execution}{NW1AE0du-2gLZR6-1}{\nwixu{NW1AE0du-26HyRR-1}\nwixd{NW1AE0du-2gLZR6-1}}}%
\nwixlogsorted{c}{{standard unix commands}{NW1AE0du-1VdKMJ-1}{\nwixu{NW1AE0du-okr1C-1}\nwixd{NW1AE0du-1VdKMJ-1}}}%
\nwixlogsorted{c}{{statistics on file contents}{NW1AE0du-1Dsqad-1}{\nwixu{NW1AE0du-1VdKMJ-1}\nwixd{NW1AE0du-1Dsqad-1}}}%
\nwixlogsorted{c}{{system-specific configuration}{NW1AE0du-45XUZU-1}{\nwixu{NW1AE0du-okr1C-1}\nwixd{NW1AE0du-45XUZU-1}}}%
\nwixlogsorted{c}{{variables for archive programs}{NW1AE0du-5Fqx6-1}{\nwixu{NW1AE0du-3TkHp6-1}\nwixd{NW1AE0du-5Fqx6-1}\nwixd{NW1AE0du-5Fqx6-2}\nwixd{NW1AE0du-5Fqx6-3}\nwixd{NW1AE0du-5Fqx6-4}\nwixd{NW1AE0du-5Fqx6-5}}}%
\nwixlogsorted{c}{{variables for compression programs}{NW1AE0du-8WF6G-1}{\nwixu{NW1AE0du-3TkHp6-1}\nwixd{NW1AE0du-8WF6G-1}\nwixd{NW1AE0du-8WF6G-2}}}%
\nwbegindocs{48}\nwdocspar
\nwenddocs{}
